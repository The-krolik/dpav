%% Generated by Sphinx.
\def\sphinxdocclass{report}
\documentclass[letterpaper,10pt,english]{sphinxmanual}
\ifdefined\pdfpxdimen
   \let\sphinxpxdimen\pdfpxdimen\else\newdimen\sphinxpxdimen
\fi \sphinxpxdimen=.75bp\relax
\ifdefined\pdfimageresolution
    \pdfimageresolution= \numexpr \dimexpr1in\relax/\sphinxpxdimen\relax
\fi
%% let collapsible pdf bookmarks panel have high depth per default
\PassOptionsToPackage{bookmarksdepth=5}{hyperref}

\PassOptionsToPackage{warn}{textcomp}
\usepackage[utf8]{inputenc}
\ifdefined\DeclareUnicodeCharacter
% support both utf8 and utf8x syntaxes
  \ifdefined\DeclareUnicodeCharacterAsOptional
    \def\sphinxDUC#1{\DeclareUnicodeCharacter{"#1}}
  \else
    \let\sphinxDUC\DeclareUnicodeCharacter
  \fi
  \sphinxDUC{00A0}{\nobreakspace}
  \sphinxDUC{2500}{\sphinxunichar{2500}}
  \sphinxDUC{2502}{\sphinxunichar{2502}}
  \sphinxDUC{2514}{\sphinxunichar{2514}}
  \sphinxDUC{251C}{\sphinxunichar{251C}}
  \sphinxDUC{2572}{\textbackslash}
\fi
\usepackage{cmap}
\usepackage[T1]{fontenc}
\usepackage{amsmath,amssymb,amstext}
\usepackage{babel}



\usepackage{tgtermes}
\usepackage{tgheros}
\renewcommand{\ttdefault}{txtt}



\usepackage[Bjarne]{fncychap}
\usepackage{sphinx}

\fvset{fontsize=auto}
\usepackage{geometry}


% Include hyperref last.
\usepackage{hyperref}
% Fix anchor placement for figures with captions.
\usepackage{hypcap}% it must be loaded after hyperref.
% Set up styles of URL: it should be placed after hyperref.
\urlstyle{same}

\addto\captionsenglish{\renewcommand{\contentsname}{Contents:}}

\usepackage{sphinxmessages}
\setcounter{tocdepth}{1}



\title{Direct Python Audio/Video}
\date{May 01, 2022}
\release{0.0.1}
\author{Vibrant Labs}
\newcommand{\sphinxlogo}{\vbox{}}
\renewcommand{\releasename}{Release}
\makeindex
\begin{document}

\pagestyle{empty}
\sphinxmaketitle
\pagestyle{plain}
\sphinxtableofcontents
\pagestyle{normal}
\phantomsection\label{\detokenize{index::doc}}


\sphinxstepscope


\chapter{dpav}
\label{\detokenize{modules:dpav}}\label{\detokenize{modules::doc}}
\sphinxstepscope


\section{dpav package}
\label{\detokenize{dpav:dpav-package}}\label{\detokenize{dpav::doc}}

\subsection{Submodules}
\label{\detokenize{dpav:submodules}}

\subsection{dpav.audio module}
\label{\detokenize{dpav:module-dpav.audio}}\label{\detokenize{dpav:dpav-audio-module}}\index{module@\spxentry{module}!dpav.audio@\spxentry{dpav.audio}}\index{dpav.audio@\spxentry{dpav.audio}!module@\spxentry{module}}\index{Audio (class in dpav.audio)@\spxentry{Audio}\spxextra{class in dpav.audio}}

\begin{fulllineitems}
\phantomsection\label{\detokenize{dpav:dpav.audio.Audio}}
\pysigstartsignatures
\pysigline{\sphinxbfcode{\sphinxupquote{class\DUrole{w}{  }}}\sphinxcode{\sphinxupquote{dpav.audio.}}\sphinxbfcode{\sphinxupquote{Audio}}}
\pysigstopsignatures
\sphinxAtStartPar
Bases: \sphinxcode{\sphinxupquote{object}}

\sphinxAtStartPar
Handles Audio capabilities of Python Direct Platform.
\begin{description}
\item[{Functions:}] \leavevmode\begin{description}
\item[{Constructor:}] \leavevmode
\sphinxAtStartPar
\_\_init\_\_()

\item[{Functions:}] \leavevmode\begin{description}
\item[{play\_sound(Hz, length)}] \leavevmode\begin{description}
\item[{If audio buffer is set:}] \leavevmode
\sphinxAtStartPar
play\_sound()

\end{description}

\end{description}

\sphinxAtStartPar
play\_sample(string\_name\_of\_wav\_file)

\item[{Setters:}] \leavevmode
\sphinxAtStartPar
set\_audio\_buffer(numpyarray)
set\_audio\_device(int)
set\_waveform(waveform)

\item[{Getters:}] \leavevmode
\sphinxAtStartPar
get\_bit\_number()\sphinxhyphen{}\textgreater{}int
get\_sample\_rate()\sphinxhyphen{}\textgreater{}int
get\_audio\_buffer()
get\_audio\_device()\sphinxhyphen{}\textgreater{}Returns int corresponding to audio device

\item[{Misc:}] \leavevmode
\sphinxAtStartPar
list\_audio\_devices()
wait\_for\_sound\_end()

\end{description}

\end{description}
\index{get\_audio\_buffer() (dpav.audio.Audio method)@\spxentry{get\_audio\_buffer()}\spxextra{dpav.audio.Audio method}}

\begin{fulllineitems}
\phantomsection\label{\detokenize{dpav:dpav.audio.Audio.get_audio_buffer}}
\pysigstartsignatures
\pysiglinewithargsret{\sphinxbfcode{\sphinxupquote{get\_audio\_buffer}}}{}{}
\pysigstopsignatures
\sphinxAtStartPar
Returns the audio buffer of the Audio class
\begin{description}
\item[{Description:}] \leavevmode
\sphinxAtStartPar
This will return none if the audio buffer has not been set by the set\_audio\_buffer method.

\sphinxAtStartPar
audioobject.get\_audio\_buffer()

\end{description}
\begin{quote}\begin{description}
\item[{Parameters}] \leavevmode
\sphinxAtStartPar
\sphinxstyleliteralstrong{\sphinxupquote{None}} \textendash{} 

\item[{Returns}] \leavevmode
\sphinxAtStartPar
numpy array

\item[{Return type}] \leavevmode
\sphinxAtStartPar
self.\_audio\_buffer

\end{description}\end{quote}

\end{fulllineitems}

\index{get\_audio\_device() (dpav.audio.Audio method)@\spxentry{get\_audio\_device()}\spxextra{dpav.audio.Audio method}}

\begin{fulllineitems}
\phantomsection\label{\detokenize{dpav:dpav.audio.Audio.get_audio_device}}
\pysigstartsignatures
\pysiglinewithargsret{\sphinxbfcode{\sphinxupquote{get\_audio\_device}}}{}{{ $\rightarrow$ int}}
\pysigstopsignatures
\sphinxAtStartPar
Gets the current audio device number of the Audio Class
\begin{description}
\item[{Description:}] \leavevmode
\sphinxAtStartPar
Assuming audioobject.set\_audio\_device(2) is called,
audioobject.get\_audio\_device() would return 2 {[}index of audio device in audioobject.list\_audio\_devices(){]}

\end{description}
\begin{quote}\begin{description}
\item[{Parameters}] \leavevmode
\sphinxAtStartPar
\sphinxstyleliteralstrong{\sphinxupquote{None}} \textendash{} 

\end{description}\end{quote}
\begin{description}
\item[{Returns}] \leavevmode
\sphinxAtStartPar
self.\_audio\_device: int value

\end{description}
\subsubsection*{Notes}

\sphinxAtStartPar
Returns the integer value of the device not the device name

\end{fulllineitems}

\index{get\_bit\_number() (dpav.audio.Audio method)@\spxentry{get\_bit\_number()}\spxextra{dpav.audio.Audio method}}

\begin{fulllineitems}
\phantomsection\label{\detokenize{dpav:dpav.audio.Audio.get_bit_number}}
\pysigstartsignatures
\pysiglinewithargsret{\sphinxbfcode{\sphinxupquote{get\_bit\_number}}}{}{{ $\rightarrow$ int}}
\pysigstopsignatures
\sphinxAtStartPar
Gets the bit rate of the Audio class
\begin{description}
\item[{Description:}] \leavevmode
\sphinxAtStartPar
Bit rate currently locked to 16 bits

\end{description}
\begin{quote}\begin{description}
\item[{Parameters}] \leavevmode
\sphinxAtStartPar
\sphinxstyleliteralstrong{\sphinxupquote{None}} \textendash{} 

\item[{Returns}] \leavevmode
\sphinxAtStartPar
The bit rate of the Audio class \sphinxhyphen{} int value

\item[{Return type}] \leavevmode
\sphinxAtStartPar
self.\_bit\_number

\end{description}\end{quote}

\end{fulllineitems}

\index{get\_sample\_rate() (dpav.audio.Audio method)@\spxentry{get\_sample\_rate()}\spxextra{dpav.audio.Audio method}}

\begin{fulllineitems}
\phantomsection\label{\detokenize{dpav:dpav.audio.Audio.get_sample_rate}}
\pysigstartsignatures
\pysiglinewithargsret{\sphinxbfcode{\sphinxupquote{get\_sample\_rate}}}{}{{ $\rightarrow$ int}}
\pysigstopsignatures
\sphinxAtStartPar
Gets the sample rate of the Audio class.
\begin{description}
\item[{Description:}] \leavevmode
\sphinxAtStartPar
Sample rate is currently locked to 44100

\end{description}
\begin{quote}\begin{description}
\item[{Parameters}] \leavevmode
\sphinxAtStartPar
\sphinxstyleliteralstrong{\sphinxupquote{None}} \textendash{} 

\item[{Returns}] \leavevmode
\sphinxAtStartPar
The sample rate of the audioClass \sphinxhyphen{} int value

\item[{Return type}] \leavevmode
\sphinxAtStartPar
self.\_sample\_rate

\end{description}\end{quote}

\end{fulllineitems}

\index{list\_audio\_devices() (dpav.audio.Audio method)@\spxentry{list\_audio\_devices()}\spxextra{dpav.audio.Audio method}}

\begin{fulllineitems}
\phantomsection\label{\detokenize{dpav:dpav.audio.Audio.list_audio_devices}}
\pysigstartsignatures
\pysiglinewithargsret{\sphinxbfcode{\sphinxupquote{list\_audio\_devices}}}{}{{ $\rightarrow$ None}}
\pysigstopsignatures
\sphinxAtStartPar
Lists the output devices on your system and adds to list self.\_devices
\begin{description}
\item[{Description:}] \leavevmode
\sphinxAtStartPar
Run this function before using set\_audio\_device() to add devices to the list devices

\sphinxAtStartPar
audioobject.list\_audio\_devices()
0 Speakers (Realtek(R) Audio)
1 VGA248 (2\sphinxhyphen{}NVIDIA High Def Audio)
2 Speakers (HyperX Cloud II Wireless)

\end{description}
\begin{quote}\begin{description}
\item[{Parameters}] \leavevmode
\sphinxAtStartPar
\sphinxstyleliteralstrong{\sphinxupquote{None}} \textendash{} 

\item[{Returns}] \leavevmode
\sphinxAtStartPar
None

\end{description}\end{quote}

\end{fulllineitems}

\index{play\_sample() (dpav.audio.Audio method)@\spxentry{play\_sample()}\spxextra{dpav.audio.Audio method}}

\begin{fulllineitems}
\phantomsection\label{\detokenize{dpav:dpav.audio.Audio.play_sample}}
\pysigstartsignatures
\pysiglinewithargsret{\sphinxbfcode{\sphinxupquote{play\_sample}}}{\emph{\DUrole{n}{sample\_name}\DUrole{p}{:}\DUrole{w}{  }\DUrole{n}{str}}}{{ $\rightarrow$ None}}
\pysigstopsignatures
\sphinxAtStartPar
Plays sounds that are wav, ogg or mp3 files.
\begin{description}
\item[{Description:}] \leavevmode
\sphinxAtStartPar
audioobject.play\_sample(mypath.mp3) would play sounds from the file mypath.mp3

\end{description}
\begin{quote}\begin{description}
\item[{Parameters}] \leavevmode
\sphinxAtStartPar
\sphinxstyleliteralstrong{\sphinxupquote{sample\_name}} \textendash{} String path or name of sound

\item[{Returns}] \leavevmode
\sphinxAtStartPar
None

\end{description}\end{quote}

\end{fulllineitems}

\index{play\_sound() (dpav.audio.Audio method)@\spxentry{play\_sound()}\spxextra{dpav.audio.Audio method}}

\begin{fulllineitems}
\phantomsection\label{\detokenize{dpav:dpav.audio.Audio.play_sound}}
\pysigstartsignatures
\pysiglinewithargsret{\sphinxbfcode{\sphinxupquote{play\_sound}}}{\emph{\DUrole{n}{input\_frequency}\DUrole{o}{=}\DUrole{default_value}{0}}, \emph{\DUrole{n}{input\_duration}\DUrole{o}{=}\DUrole{default_value}{0}}}{{ $\rightarrow$ None}}
\pysigstopsignatures
\sphinxAtStartPar
Primary sound playing method of the audio class.
\begin{description}
\item[{Description:}] \leavevmode
\sphinxAtStartPar
Play sounds directly from this function
Need to run set\_audio\_device() or will default to the default audio device
You can use set\_waveform to change the type.
play\_sound is somewhat overloaded to where if you have an audioBuffer set using set\_audio\_buffer, you can call play\_sound()
\begin{quote}

\sphinxAtStartPar
and it will play whatever that audio\_buffer is e.g. wav files
Example in examples/custombuffer.py
\end{quote}

\sphinxAtStartPar
play\_sound(440, 1) would play an A note for one second with the sin waveform set.

\end{description}
\begin{quote}\begin{description}
\item[{Parameters}] \leavevmode\begin{itemize}
\item {} 
\sphinxAtStartPar
\sphinxstyleliteralstrong{\sphinxupquote{input\_frequency}} \textendash{} int value \sphinxhyphen{} input frequency in Hz

\item {} 
\sphinxAtStartPar
\sphinxstyleliteralstrong{\sphinxupquote{input\_duration}} \textendash{} int value \sphinxhyphen{} duration in seconds

\end{itemize}

\item[{Raises}] \leavevmode
\sphinxAtStartPar
\sphinxstyleliteralstrong{\sphinxupquote{TypeError}} \textendash{} If input\_duration not a number, or \textless{} 0

\item[{Returns}] \leavevmode
\sphinxAtStartPar
None

\end{description}\end{quote}

\end{fulllineitems}

\index{set\_audio\_buffer() (dpav.audio.Audio method)@\spxentry{set\_audio\_buffer()}\spxextra{dpav.audio.Audio method}}

\begin{fulllineitems}
\phantomsection\label{\detokenize{dpav:dpav.audio.Audio.set_audio_buffer}}
\pysigstartsignatures
\pysiglinewithargsret{\sphinxbfcode{\sphinxupquote{set\_audio\_buffer}}}{\emph{\DUrole{n}{ab}}}{{ $\rightarrow$ None}}
\pysigstopsignatures
\sphinxAtStartPar
Sets the audio buffer of the Audio Class.
\begin{description}
\item[{Description:}] \leavevmode
\sphinxAtStartPar
The audio buffer needs to have two rows so that way stereo works as intended.
You can set the audio buffer to wav file data by fetching numpy arrays using wav or scipy,
however only 16 bit waves are supported. This process can be seen in custom\_buffer.py w/ the
utility function sixteenWavtoRawData
\begin{description}
\item[{Examples:}] \leavevmode
\sphinxAtStartPar
\# 44100 = sample rate
\# 32767 is 2 \textasciicircum{} (our bit depth \sphinxhyphen{}1)\sphinxhyphen{}1 and is essentially the number of samples per time stamp
\# 260 and 290 are our tones in hz
\# Below generates a buffer 1 second long of sin wave data\sphinxhyphen{}identical to the method used in house
data = numpy.zeros((44100, 2), dtype=numpy.int16)
for s in range(44100):
\begin{quote}

\sphinxAtStartPar
t = float(s) / 44100
data{[}s{]}{[}0{]} = int(round(32767 * math.sin(2 * math.pi * 260 * t)))
data{[}s{]}{[}1{]} = int(round(32767 * math.sin(2 * math.pi * 290 * t)))
\end{quote}

\sphinxAtStartPar
audioobject.set\_audio\_buffer(data)

\end{description}

\end{description}
\begin{quote}\begin{description}
\item[{Parameters}] \leavevmode
\sphinxAtStartPar
\sphinxstyleliteralstrong{\sphinxupquote{ab}} \textendash{} numpy array of shape(samples, channels) e.g. ab{[}44100{]}{[}2{]}

\item[{Returns}] \leavevmode
\sphinxAtStartPar
None

\end{description}\end{quote}

\end{fulllineitems}

\index{set\_audio\_device() (dpav.audio.Audio method)@\spxentry{set\_audio\_device()}\spxextra{dpav.audio.Audio method}}

\begin{fulllineitems}
\phantomsection\label{\detokenize{dpav:dpav.audio.Audio.set_audio_device}}
\pysigstartsignatures
\pysiglinewithargsret{\sphinxbfcode{\sphinxupquote{set\_audio\_device}}}{\emph{\DUrole{n}{device}\DUrole{p}{:}\DUrole{w}{  }\DUrole{n}{int}}}{{ $\rightarrow$ int}}
\pysigstopsignatures
\sphinxAtStartPar
Sets the current audio device of the Audio class.
\begin{description}
\item[{Description:}] \leavevmode
\sphinxAtStartPar
This can only be set ONCE per instance. To change devices, del the current instance
set the new device, and continue
This needs to be run after list\_audio\_device() in order to see list of audio devices
If not run the device will default to the current device being used by the machine

\sphinxAtStartPar
audioobject.set\_audio\_device(2)
Based on example in list\_audio\_devices() this would change the device to Speakers (HyperX Cloud II Wireless)

\end{description}
\begin{quote}\begin{description}
\item[{Parameters}] \leavevmode
\sphinxAtStartPar
\sphinxstyleliteralstrong{\sphinxupquote{device}} \textendash{} int value \sphinxhyphen{} see all int values for each device by running list\_audio\_devices()

\item[{Returns}] \leavevmode
\sphinxAtStartPar
None

\end{description}\end{quote}

\end{fulllineitems}

\index{set\_waveform() (dpav.audio.Audio method)@\spxentry{set\_waveform()}\spxextra{dpav.audio.Audio method}}

\begin{fulllineitems}
\phantomsection\label{\detokenize{dpav:dpav.audio.Audio.set_waveform}}
\pysigstartsignatures
\pysiglinewithargsret{\sphinxbfcode{\sphinxupquote{set\_waveform}}}{\emph{\DUrole{n}{wave}}}{{ $\rightarrow$ None}}
\pysigstopsignatures
\sphinxAtStartPar
Sets the expression governing the wave form playing
\begin{description}
\item[{Description:}] \leavevmode
\sphinxAtStartPar
play\_audio uses this in buffer generation

\sphinxAtStartPar
audioobject.set\_waveform(object.wave\_table.sin)
This would change to the waveform sin contained in the wave\_table class
The wave functions need to take in a input frequency as well as a timestep parameter
to solve for a particular frequency at a given time step. See wave\_table for an example of this.

\end{description}
\begin{quote}\begin{description}
\item[{Parameters}] \leavevmode
\sphinxAtStartPar
\sphinxstyleliteralstrong{\sphinxupquote{Wave}} \textendash{} takes a mathematical expression function ‘pointer’ in the form of f(inputfreq, timestep)

\item[{Returns}] \leavevmode
\sphinxAtStartPar
None

\end{description}\end{quote}

\end{fulllineitems}

\index{wait\_for\_sound\_end() (dpav.audio.Audio method)@\spxentry{wait\_for\_sound\_end()}\spxextra{dpav.audio.Audio method}}

\begin{fulllineitems}
\phantomsection\label{\detokenize{dpav:dpav.audio.Audio.wait_for_sound_end}}
\pysigstartsignatures
\pysiglinewithargsret{\sphinxbfcode{\sphinxupquote{wait\_for\_sound\_end}}}{}{}
\pysigstopsignatures
\sphinxAtStartPar
Function call that is placed at the end of scripts without a pygame window instance so sounds play to their full duration without a
\begin{description}
\item[{Description:}] \leavevmode
\sphinxAtStartPar
Placed at the end of python files that do not have loops. Otherwise, sounds would be cut off prematurely.
\begin{description}
\item[{Example:}] \leavevmode
\sphinxAtStartPar
play\_sound(440, 10)
wait\_for\_sound\_end() \# This prevents the process from closing out before the sound ends.

\end{description}

\end{description}
\begin{quote}\begin{description}
\item[{Parameters}] \leavevmode
\sphinxAtStartPar
\sphinxstyleliteralstrong{\sphinxupquote{None}} \textendash{} 

\item[{Returns}] \leavevmode
\sphinxAtStartPar
None

\end{description}\end{quote}

\sphinxAtStartPar
Notes:

\end{fulllineitems}


\end{fulllineitems}

\index{wave\_table (class in dpav.audio)@\spxentry{wave\_table}\spxextra{class in dpav.audio}}

\begin{fulllineitems}
\phantomsection\label{\detokenize{dpav:dpav.audio.wave_table}}
\pysigstartsignatures
\pysigline{\sphinxbfcode{\sphinxupquote{class\DUrole{w}{  }}}\sphinxcode{\sphinxupquote{dpav.audio.}}\sphinxbfcode{\sphinxupquote{wave\_table}}}
\pysigstopsignatures
\sphinxAtStartPar
Bases: \sphinxcode{\sphinxupquote{object}}

\sphinxAtStartPar
This is a class holding waveforms for usage with the play\_sound method.
\begin{description}
\item[{There are 5 waveforms:}] \leavevmode
\sphinxAtStartPar
sin
saw
square
noise
triangle

\end{description}
\subsubsection*{Example}

\sphinxAtStartPar
waves = wave\_table()
sinefunc = waves.sin
\index{noise() (dpav.audio.wave\_table method)@\spxentry{noise()}\spxextra{dpav.audio.wave\_table method}}

\begin{fulllineitems}
\phantomsection\label{\detokenize{dpav:dpav.audio.wave_table.noise}}
\pysigstartsignatures
\pysiglinewithargsret{\sphinxbfcode{\sphinxupquote{noise}}}{\emph{\DUrole{n}{input\_frequency}}, \emph{\DUrole{n}{t}}}{}
\pysigstopsignatures
\sphinxAtStartPar
Random white noise
\begin{description}
\item[{Description:}] \leavevmode
\sphinxAtStartPar
Warning: VERY LOUD

\end{description}
\begin{quote}\begin{description}
\item[{Parameters}] \leavevmode\begin{itemize}
\item {} 
\sphinxAtStartPar
\sphinxstyleliteralstrong{\sphinxupquote{input\_frequency}} \textendash{} 

\item {} 
\sphinxAtStartPar
\sphinxstyleliteralstrong{\sphinxupquote{t}} \textendash{} 

\end{itemize}

\item[{Return type}] \leavevmode
\sphinxAtStartPar
random.random() * input\_frequency * t

\end{description}\end{quote}

\end{fulllineitems}

\index{saw() (dpav.audio.wave\_table method)@\spxentry{saw()}\spxextra{dpav.audio.wave\_table method}}

\begin{fulllineitems}
\phantomsection\label{\detokenize{dpav:dpav.audio.wave_table.saw}}
\pysigstartsignatures
\pysiglinewithargsret{\sphinxbfcode{\sphinxupquote{saw}}}{\emph{\DUrole{n}{input\_frequency}}, \emph{\DUrole{n}{t}}}{}
\pysigstopsignatures
\sphinxAtStartPar
Saw wave
\begin{quote}\begin{description}
\item[{Parameters}] \leavevmode\begin{itemize}
\item {} 
\sphinxAtStartPar
\sphinxstyleliteralstrong{\sphinxupquote{input\_frequency}} \textendash{} 

\item {} 
\sphinxAtStartPar
\sphinxstyleliteralstrong{\sphinxupquote{t}} \textendash{} 

\end{itemize}

\item[{Return type}] \leavevmode
\sphinxAtStartPar
t * input\_frequency \sphinxhyphen{} math.floor(t * input\_frequency)

\end{description}\end{quote}

\end{fulllineitems}

\index{sin() (dpav.audio.wave\_table method)@\spxentry{sin()}\spxextra{dpav.audio.wave\_table method}}

\begin{fulllineitems}
\phantomsection\label{\detokenize{dpav:dpav.audio.wave_table.sin}}
\pysigstartsignatures
\pysiglinewithargsret{\sphinxbfcode{\sphinxupquote{sin}}}{\emph{\DUrole{n}{input\_frequency}}, \emph{\DUrole{n}{t}}}{}
\pysigstopsignatures
\sphinxAtStartPar
Sin wave form, default for libary
\begin{quote}\begin{description}
\item[{Parameters}] \leavevmode\begin{itemize}
\item {} 
\sphinxAtStartPar
\sphinxstyleliteralstrong{\sphinxupquote{input\_frequency}} \textendash{} 

\item {} 
\sphinxAtStartPar
\sphinxstyleliteralstrong{\sphinxupquote{t}} \textendash{} 

\end{itemize}

\item[{Return type}] \leavevmode
\sphinxAtStartPar
math.sin(2 * math.pi * input\_frequency * t)

\end{description}\end{quote}

\end{fulllineitems}

\index{square() (dpav.audio.wave\_table method)@\spxentry{square()}\spxextra{dpav.audio.wave\_table method}}

\begin{fulllineitems}
\phantomsection\label{\detokenize{dpav:dpav.audio.wave_table.square}}
\pysigstartsignatures
\pysiglinewithargsret{\sphinxbfcode{\sphinxupquote{square}}}{\emph{\DUrole{n}{input\_frequency}}, \emph{\DUrole{n}{t}}}{}
\pysigstopsignatures
\sphinxAtStartPar
Square wave form
\begin{quote}\begin{description}
\item[{Parameters}] \leavevmode\begin{itemize}
\item {} 
\sphinxAtStartPar
\sphinxstyleliteralstrong{\sphinxupquote{input\_frequency}} \textendash{} 

\item {} 
\sphinxAtStartPar
\sphinxstyleliteralstrong{\sphinxupquote{t}} \textendash{} 

\end{itemize}

\item[{Return type}] \leavevmode
\sphinxAtStartPar
round(math.sin(2 * math.pi * input\_frequency * t))

\end{description}\end{quote}

\end{fulllineitems}

\index{triangle() (dpav.audio.wave\_table method)@\spxentry{triangle()}\spxextra{dpav.audio.wave\_table method}}

\begin{fulllineitems}
\phantomsection\label{\detokenize{dpav:dpav.audio.wave_table.triangle}}
\pysigstartsignatures
\pysiglinewithargsret{\sphinxbfcode{\sphinxupquote{triangle}}}{\emph{\DUrole{n}{input\_frequency}}, \emph{\DUrole{n}{t}}}{}
\pysigstopsignatures
\sphinxAtStartPar
Triangle wave, similar in sound to saw + sin together
\begin{quote}\begin{description}
\item[{Parameters}] \leavevmode\begin{itemize}
\item {} 
\sphinxAtStartPar
\sphinxstyleliteralstrong{\sphinxupquote{input\_frequency}} \textendash{} 

\item {} 
\sphinxAtStartPar
\sphinxstyleliteralstrong{\sphinxupquote{t}} \textendash{} 

\end{itemize}

\item[{Return type}] \leavevmode
\sphinxAtStartPar
2 * abs((t * input\_frequency) / 1 \sphinxhyphen{} math.floor(((t * input\_frequency) / 1) + 0.5))

\end{description}\end{quote}

\end{fulllineitems}


\end{fulllineitems}



\subsection{dpav.utility module}
\label{\detokenize{dpav:module-dpav.utility}}\label{\detokenize{dpav:dpav-utility-module}}\index{module@\spxentry{module}!dpav.utility@\spxentry{dpav.utility}}\index{dpav.utility@\spxentry{dpav.utility}!module@\spxentry{module}}\index{draw\_circle() (in module dpav.utility)@\spxentry{draw\_circle()}\spxextra{in module dpav.utility}}

\begin{fulllineitems}
\phantomsection\label{\detokenize{dpav:dpav.utility.draw_circle}}
\pysigstartsignatures
\pysiglinewithargsret{\sphinxcode{\sphinxupquote{dpav.utility.}}\sphinxbfcode{\sphinxupquote{draw\_circle}}}{\emph{\DUrole{n}{vb}\DUrole{p}{:}\DUrole{w}{  }\DUrole{n}{{\hyperref[\detokenize{dpav:dpav.vbuffer.VBuffer}]{\sphinxcrossref{dpav.vbuffer.VBuffer}}}}}, \emph{\DUrole{n}{center}\DUrole{p}{:}\DUrole{w}{  }\DUrole{n}{list}}, \emph{\DUrole{n}{r}\DUrole{p}{:}\DUrole{w}{  }\DUrole{n}{float}}, \emph{\DUrole{n}{color}\DUrole{p}{:}\DUrole{w}{  }\DUrole{n}{int}}}{}
\pysigstopsignatures
\sphinxAtStartPar
Draws a circle onto a visual buffer of a specified color and radius
around a given center point using Bresenham’s algorithm.

\end{fulllineitems}

\index{draw\_line() (in module dpav.utility)@\spxentry{draw\_line()}\spxextra{in module dpav.utility}}

\begin{fulllineitems}
\phantomsection\label{\detokenize{dpav:dpav.utility.draw_line}}
\pysigstartsignatures
\pysiglinewithargsret{\sphinxcode{\sphinxupquote{dpav.utility.}}\sphinxbfcode{\sphinxupquote{draw\_line}}}{\emph{\DUrole{n}{vb}\DUrole{p}{:}\DUrole{w}{  }\DUrole{n}{{\hyperref[\detokenize{dpav:dpav.vbuffer.VBuffer}]{\sphinxcrossref{dpav.vbuffer.VBuffer}}}}}, \emph{\DUrole{n}{p0}\DUrole{p}{:}\DUrole{w}{  }\DUrole{n}{list}}, \emph{\DUrole{n}{p1}\DUrole{p}{:}\DUrole{w}{  }\DUrole{n}{list}}, \emph{\DUrole{n}{color}\DUrole{p}{:}\DUrole{w}{  }\DUrole{n}{int}}}{}
\pysigstopsignatures
\sphinxAtStartPar
Draws a line on a visual buffer from p0 to p1 using Bresenham’s algorithm

\end{fulllineitems}

\index{draw\_polygon() (in module dpav.utility)@\spxentry{draw\_polygon()}\spxextra{in module dpav.utility}}

\begin{fulllineitems}
\phantomsection\label{\detokenize{dpav:dpav.utility.draw_polygon}}
\pysigstartsignatures
\pysiglinewithargsret{\sphinxcode{\sphinxupquote{dpav.utility.}}\sphinxbfcode{\sphinxupquote{draw\_polygon}}}{\emph{\DUrole{n}{vb}\DUrole{p}{:}\DUrole{w}{  }\DUrole{n}{{\hyperref[\detokenize{dpav:dpav.vbuffer.VBuffer}]{\sphinxcrossref{dpav.vbuffer.VBuffer}}}}}, \emph{\DUrole{n}{vertices}\DUrole{p}{:}\DUrole{w}{  }\DUrole{n}{list}}, \emph{\DUrole{n}{color}\DUrole{p}{:}\DUrole{w}{  }\DUrole{n}{int}}}{}
\pysigstopsignatures
\sphinxAtStartPar
Draws a polygon in a visual buffer with the given vertices utilizing the
order in which they are given.

\end{fulllineitems}

\index{draw\_rectangle() (in module dpav.utility)@\spxentry{draw\_rectangle()}\spxextra{in module dpav.utility}}

\begin{fulllineitems}
\phantomsection\label{\detokenize{dpav:dpav.utility.draw_rectangle}}
\pysigstartsignatures
\pysiglinewithargsret{\sphinxcode{\sphinxupquote{dpav.utility.}}\sphinxbfcode{\sphinxupquote{draw\_rectangle}}}{\emph{\DUrole{n}{vbuffer}}, \emph{\DUrole{n}{color}}, \emph{\DUrole{n}{pt1}}, \emph{\DUrole{n}{pt2}}}{}
\pysigstopsignatures
\end{fulllineitems}

\index{fill() (in module dpav.utility)@\spxentry{fill()}\spxextra{in module dpav.utility}}

\begin{fulllineitems}
\phantomsection\label{\detokenize{dpav:dpav.utility.fill}}
\pysigstartsignatures
\pysiglinewithargsret{\sphinxcode{\sphinxupquote{dpav.utility.}}\sphinxbfcode{\sphinxupquote{fill}}}{\emph{\DUrole{n}{vb}\DUrole{p}{:}\DUrole{w}{  }\DUrole{n}{{\hyperref[\detokenize{dpav:dpav.vbuffer.VBuffer}]{\sphinxcrossref{dpav.vbuffer.VBuffer}}}}}, \emph{\DUrole{n}{color}\DUrole{p}{:}\DUrole{w}{  }\DUrole{n}{int}}, \emph{\DUrole{n}{vertices}}}{}
\pysigstopsignatures
\sphinxAtStartPar
Fills a polygon defined by a set of vertices with a color.

\end{fulllineitems}

\index{flip\_horizontally() (in module dpav.utility)@\spxentry{flip\_horizontally()}\spxextra{in module dpav.utility}}

\begin{fulllineitems}
\phantomsection\label{\detokenize{dpav:dpav.utility.flip_horizontally}}
\pysigstartsignatures
\pysiglinewithargsret{\sphinxcode{\sphinxupquote{dpav.utility.}}\sphinxbfcode{\sphinxupquote{flip\_horizontally}}}{\emph{\DUrole{n}{vb}\DUrole{p}{:}\DUrole{w}{  }\DUrole{n}{{\hyperref[\detokenize{dpav:dpav.vbuffer.VBuffer}]{\sphinxcrossref{dpav.vbuffer.VBuffer}}}}}}{{ $\rightarrow$ {\hyperref[\detokenize{dpav:dpav.vbuffer.VBuffer}]{\sphinxcrossref{dpav.vbuffer.VBuffer}}}}}
\pysigstopsignatures
\end{fulllineitems}

\index{flip\_vertically() (in module dpav.utility)@\spxentry{flip\_vertically()}\spxextra{in module dpav.utility}}

\begin{fulllineitems}
\phantomsection\label{\detokenize{dpav:dpav.utility.flip_vertically}}
\pysigstartsignatures
\pysiglinewithargsret{\sphinxcode{\sphinxupquote{dpav.utility.}}\sphinxbfcode{\sphinxupquote{flip\_vertically}}}{\emph{\DUrole{n}{vb}\DUrole{p}{:}\DUrole{w}{  }\DUrole{n}{{\hyperref[\detokenize{dpav:dpav.vbuffer.VBuffer}]{\sphinxcrossref{dpav.vbuffer.VBuffer}}}}}}{{ $\rightarrow$ {\hyperref[\detokenize{dpav:dpav.vbuffer.VBuffer}]{\sphinxcrossref{dpav.vbuffer.VBuffer}}}}}
\pysigstopsignatures
\end{fulllineitems}

\index{get\_note\_from\_string() (in module dpav.utility)@\spxentry{get\_note\_from\_string()}\spxextra{in module dpav.utility}}

\begin{fulllineitems}
\phantomsection\label{\detokenize{dpav:dpav.utility.get_note_from_string}}
\pysigstartsignatures
\pysiglinewithargsret{\sphinxcode{\sphinxupquote{dpav.utility.}}\sphinxbfcode{\sphinxupquote{get\_note\_from\_string}}}{\emph{\DUrole{n}{string}}, \emph{\DUrole{n}{octave}}}{}
\pysigstopsignatures
\sphinxAtStartPar
Given a string representing a note, this will return a hz
IN: string representing the note e.g. Ab, C, E\#
OUT: returns hz

\end{fulllineitems}

\index{load\_image() (in module dpav.utility)@\spxentry{load\_image()}\spxextra{in module dpav.utility}}

\begin{fulllineitems}
\phantomsection\label{\detokenize{dpav:dpav.utility.load_image}}
\pysigstartsignatures
\pysiglinewithargsret{\sphinxcode{\sphinxupquote{dpav.utility.}}\sphinxbfcode{\sphinxupquote{load\_image}}}{\emph{\DUrole{n}{filepath}}}{{ $\rightarrow$ numpy.ndarray}}
\pysigstopsignatures\begin{description}
\item[{Description:}] \leavevmode
\sphinxAtStartPar
Takes the file path of an image and returns an np.ndarray in hex

\end{description}

\end{fulllineitems}

\index{point\_in\_polygon() (in module dpav.utility)@\spxentry{point\_in\_polygon()}\spxextra{in module dpav.utility}}

\begin{fulllineitems}
\phantomsection\label{\detokenize{dpav:dpav.utility.point_in_polygon}}
\pysigstartsignatures
\pysiglinewithargsret{\sphinxcode{\sphinxupquote{dpav.utility.}}\sphinxbfcode{\sphinxupquote{point\_in\_polygon}}}{\emph{\DUrole{n}{x}\DUrole{p}{:}\DUrole{w}{  }\DUrole{n}{int}}, \emph{\DUrole{n}{y}\DUrole{p}{:}\DUrole{w}{  }\DUrole{n}{int}}, \emph{\DUrole{n}{vertices}}}{{ $\rightarrow$ bool}}
\pysigstopsignatures
\sphinxAtStartPar
Uses the Even\sphinxhyphen{}Odd Rule to determine whether or not the pixel at coordinate
(x,y) is inside the polygon defined by a set of vertices.

\end{fulllineitems}

\index{replace\_color() (in module dpav.utility)@\spxentry{replace\_color()}\spxextra{in module dpav.utility}}

\begin{fulllineitems}
\phantomsection\label{\detokenize{dpav:dpav.utility.replace_color}}
\pysigstartsignatures
\pysiglinewithargsret{\sphinxcode{\sphinxupquote{dpav.utility.}}\sphinxbfcode{\sphinxupquote{replace\_color}}}{\emph{\DUrole{n}{vb}\DUrole{p}{:}\DUrole{w}{  }\DUrole{n}{{\hyperref[\detokenize{dpav:dpav.vbuffer.VBuffer}]{\sphinxcrossref{dpav.vbuffer.VBuffer}}}}}, \emph{\DUrole{n}{replaced\_color}\DUrole{p}{:}\DUrole{w}{  }\DUrole{n}{int}}, \emph{\DUrole{n}{new\_color}\DUrole{p}{:}\DUrole{w}{  }\DUrole{n}{int}}}{}
\pysigstopsignatures
\sphinxAtStartPar
Replace all pixels of value replaced\_color with new\_color in a visual
buffer vb.

\end{fulllineitems}

\index{rgb\_to\_hex() (in module dpav.utility)@\spxentry{rgb\_to\_hex()}\spxextra{in module dpav.utility}}

\begin{fulllineitems}
\phantomsection\label{\detokenize{dpav:dpav.utility.rgb_to_hex}}
\pysigstartsignatures
\pysiglinewithargsret{\sphinxcode{\sphinxupquote{dpav.utility.}}\sphinxbfcode{\sphinxupquote{rgb\_to\_hex}}}{\emph{\DUrole{n}{arr}}}{}
\pysigstopsignatures\begin{description}
\item[{Description:}] \leavevmode
\sphinxAtStartPar
Takes an np.ndarray in rgb and returns it in hex

\end{description}

\end{fulllineitems}

\index{sixteenWavtoRawData() (in module dpav.utility)@\spxentry{sixteenWavtoRawData()}\spxextra{in module dpav.utility}}

\begin{fulllineitems}
\phantomsection\label{\detokenize{dpav:dpav.utility.sixteenWavtoRawData}}
\pysigstartsignatures
\pysiglinewithargsret{\sphinxcode{\sphinxupquote{dpav.utility.}}\sphinxbfcode{\sphinxupquote{sixteenWavtoRawData}}}{\emph{\DUrole{n}{wavefile}}}{}
\pysigstopsignatures
\sphinxAtStartPar
takes in a string path/name of a wav file, converts it to numpy array

\end{fulllineitems}

\index{translate() (in module dpav.utility)@\spxentry{translate()}\spxextra{in module dpav.utility}}

\begin{fulllineitems}
\phantomsection\label{\detokenize{dpav:dpav.utility.translate}}
\pysigstartsignatures
\pysiglinewithargsret{\sphinxcode{\sphinxupquote{dpav.utility.}}\sphinxbfcode{\sphinxupquote{translate}}}{\emph{\DUrole{n}{vb}\DUrole{p}{:}\DUrole{w}{  }\DUrole{n}{{\hyperref[\detokenize{dpav:dpav.vbuffer.VBuffer}]{\sphinxcrossref{dpav.vbuffer.VBuffer}}}}}, \emph{\DUrole{n}{x\_translation}\DUrole{p}{:}\DUrole{w}{  }\DUrole{n}{int}}, \emph{\DUrole{n}{y\_translation}\DUrole{p}{:}\DUrole{w}{  }\DUrole{n}{int}}}{{ $\rightarrow$ {\hyperref[\detokenize{dpav:dpav.vbuffer.VBuffer}]{\sphinxcrossref{dpav.vbuffer.VBuffer}}}}}
\pysigstopsignatures
\end{fulllineitems}



\subsection{dpav.vbuffer module}
\label{\detokenize{dpav:module-dpav.vbuffer}}\label{\detokenize{dpav:dpav-vbuffer-module}}\index{module@\spxentry{module}!dpav.vbuffer@\spxentry{dpav.vbuffer}}\index{dpav.vbuffer@\spxentry{dpav.vbuffer}!module@\spxentry{module}}\index{VBuffer (class in dpav.vbuffer)@\spxentry{VBuffer}\spxextra{class in dpav.vbuffer}}

\begin{fulllineitems}
\phantomsection\label{\detokenize{dpav:dpav.vbuffer.VBuffer}}
\pysigstartsignatures
\pysiglinewithargsret{\sphinxbfcode{\sphinxupquote{class\DUrole{w}{  }}}\sphinxcode{\sphinxupquote{dpav.vbuffer.}}\sphinxbfcode{\sphinxupquote{VBuffer}}}{\emph{\DUrole{n}{arg1}\DUrole{p}{:}\DUrole{w}{  }\DUrole{n}{list\DUrole{w}{  }\DUrole{p}{|}\DUrole{w}{  }tuple\DUrole{w}{  }\DUrole{p}{|}\DUrole{w}{  }numpy.ndarray}\DUrole{w}{  }\DUrole{o}{=}\DUrole{w}{  }\DUrole{default_value}{(800, 600)}}}{}
\pysigstopsignatures
\sphinxAtStartPar
Bases: \sphinxcode{\sphinxupquote{object}}

\sphinxAtStartPar
Visual buffer for the Python Direct Platform

\sphinxAtStartPar
Holds a 2D array of hex color values. Each element represents a pixel,
whose coordinates are its index. VBuffer can be loaded and displayed by
the window class.
\begin{quote}\begin{description}
\item[{Parameters}] \leavevmode
\sphinxAtStartPar
\sphinxstyleliteralstrong{\sphinxupquote{arg1}} (\sphinxstyleliteralemphasis{\sphinxupquote{\{}}\sphinxstyleliteralemphasis{\sphinxupquote{(}}\sphinxstyleliteralemphasis{\sphinxupquote{int}}\sphinxstyleliteralemphasis{\sphinxupquote{, }}\sphinxstyleliteralemphasis{\sphinxupquote{int}}\sphinxstyleliteralemphasis{\sphinxupquote{)}}\sphinxstyleliteralemphasis{\sphinxupquote{|}}\sphinxstyleliteralemphasis{\sphinxupquote{np.ndarray}}\sphinxstyleliteralemphasis{\sphinxupquote{(}}\sphinxstyleliteralemphasis{\sphinxupquote{int}}\sphinxstyleliteralemphasis{\sphinxupquote{, }}\sphinxstyleliteralemphasis{\sphinxupquote{int}}\sphinxstyleliteralemphasis{\sphinxupquote{)}}\sphinxstyleliteralemphasis{\sphinxupquote{\}}}) \textendash{} 
\sphinxAtStartPar
Either array dimensions or a 2\sphinxhyphen{}dimensional numpy array of integers

\sphinxAtStartPar
If dimensions, will create zeroed\sphinxhyphen{}out 2D array of the selected
dimensions. Defaults to 800x600.

\sphinxAtStartPar
If numpy array, will set buffer to the contents of that array.


\end{description}\end{quote}


\begin{fulllineitems}

\pysigstartsignatures
\pysigline{\sphinxbfcode{\sphinxupquote{Constructor:}}}
\pysigstopsignatures
\sphinxAtStartPar
\_\_init\_\_(self, arg1=(800, 600)) \sphinxhyphen{}\textgreater{} None

\end{fulllineitems}



\begin{fulllineitems}

\pysigstartsignatures
\pysigline{\sphinxbfcode{\sphinxupquote{Overloads:}}}
\pysigstopsignatures
\sphinxAtStartPar
\_\_getitem\_\_(self, idx) \sphinxhyphen{}\textgreater{} int
\_\_setitem\_\_(self, idx, val) \sphinxhyphen{}\textgreater{} None
\_\_len\_\_(self) \sphinxhyphen{}\textgreater{} int

\end{fulllineitems}



\begin{fulllineitems}

\pysigstartsignatures
\pysigline{\sphinxbfcode{\sphinxupquote{properties:}}}
\pysigstopsignatures\begin{description}
\item[{getter:}] \leavevmode
\sphinxAtStartPar
dimensions(self) \sphinxhyphen{}\textgreater{} (int, int)

\item[{setter:}] \leavevmode
\sphinxAtStartPar
dimensions(self, val) \sphinxhyphen{}\textgreater{} None

\end{description}

\end{fulllineitems}



\begin{fulllineitems}

\pysigstartsignatures
\pysigline{\sphinxbfcode{\sphinxupquote{Setter:}}}
\pysigstopsignatures
\sphinxAtStartPar
write\_pixel(self, coords, val) \sphinxhyphen{}\textgreater{} None
set\_buffer(self, buf) \sphinxhyphen{}\textgreater{} None
clear(self) \sphinxhyphen{}\textgreater{} None
fill(self, color: int) \sphinxhyphen{}\textgreater{} None

\end{fulllineitems}



\begin{fulllineitems}

\pysigstartsignatures
\pysigline{\sphinxbfcode{\sphinxupquote{Getters:}}}
\pysigstopsignatures
\sphinxAtStartPar
get\_pixel(self, coords) \sphinxhyphen{}\textgreater{} int
get\_dimensions(self) \sphinxhyphen{}\textgreater{} (int, int)

\end{fulllineitems}



\begin{fulllineitems}

\pysigstartsignatures
\pysigline{\sphinxbfcode{\sphinxupquote{File~I/O:}}}
\pysigstopsignatures
\sphinxAtStartPar
save\_buffer\_to\_file(self, filename) \sphinxhyphen{}\textgreater{} None
load\_buffer\_from\_file(self, filename) \sphinxhyphen{}\textgreater{} None

\end{fulllineitems}



\begin{fulllineitems}

\pysigstartsignatures
\pysigline{\sphinxbfcode{\sphinxupquote{Error~Checking:}}}
\pysigstopsignatures
\sphinxAtStartPar
\_check\_numpy\_arr(self,arg1,arg\_name,method\_name) \sphinxhyphen{}\textgreater{} None
\_check\_coord\_type(self, coords, arg\_name, method\_name) \sphinxhyphen{}\textgreater{} None
\_check\_coord\_vals(self, x, y, method\_name) \sphinxhyphen{}\textgreater{} None

\end{fulllineitems}

\index{clear() (dpav.vbuffer.VBuffer method)@\spxentry{clear()}\spxextra{dpav.vbuffer.VBuffer method}}

\begin{fulllineitems}
\phantomsection\label{\detokenize{dpav:dpav.vbuffer.VBuffer.clear}}
\pysigstartsignatures
\pysiglinewithargsret{\sphinxbfcode{\sphinxupquote{clear}}}{}{{ $\rightarrow$ None}}
\pysigstopsignatures
\sphinxAtStartPar
Set every pixel in buffer to 0 (hex value for black).

\end{fulllineitems}

\index{dimensions (dpav.vbuffer.VBuffer property)@\spxentry{dimensions}\spxextra{dpav.vbuffer.VBuffer property}}

\begin{fulllineitems}
\phantomsection\label{\detokenize{dpav:dpav.vbuffer.VBuffer.dimensions}}
\pysigstartsignatures
\pysigline{\sphinxbfcode{\sphinxupquote{property\DUrole{w}{  }}}\sphinxbfcode{\sphinxupquote{dimensions}}\sphinxbfcode{\sphinxupquote{\DUrole{p}{:}\DUrole{w}{  }list\DUrole{w}{  }\DUrole{p}{|}\DUrole{w}{  }tuple}}}
\pysigstopsignatures
\sphinxAtStartPar
Return dimensions of buffer.

\end{fulllineitems}

\index{fill() (dpav.vbuffer.VBuffer method)@\spxentry{fill()}\spxextra{dpav.vbuffer.VBuffer method}}

\begin{fulllineitems}
\phantomsection\label{\detokenize{dpav:dpav.vbuffer.VBuffer.fill}}
\pysigstartsignatures
\pysiglinewithargsret{\sphinxbfcode{\sphinxupquote{fill}}}{\emph{\DUrole{n}{color}\DUrole{p}{:}\DUrole{w}{  }\DUrole{n}{int}}}{{ $\rightarrow$ None}}
\pysigstopsignatures
\sphinxAtStartPar
Set every pixel in the buffer to a given color.
\begin{quote}\begin{description}
\item[{Parameters}] \leavevmode
\sphinxAtStartPar
\sphinxstyleliteralstrong{\sphinxupquote{color}} (\sphinxstyleliteralemphasis{\sphinxupquote{Hex color code}}) \textendash{} 

\end{description}\end{quote}

\end{fulllineitems}

\index{get\_dimensions() (dpav.vbuffer.VBuffer method)@\spxentry{get\_dimensions()}\spxextra{dpav.vbuffer.VBuffer method}}

\begin{fulllineitems}
\phantomsection\label{\detokenize{dpav:dpav.vbuffer.VBuffer.get_dimensions}}
\pysigstartsignatures
\pysiglinewithargsret{\sphinxbfcode{\sphinxupquote{get\_dimensions}}}{}{{ $\rightarrow$ list\DUrole{w}{  }\DUrole{p}{|}\DUrole{w}{  }tuple}}
\pysigstopsignatures
\sphinxAtStartPar
Return dimensions of visual buffer array.

\end{fulllineitems}

\index{get\_pixel() (dpav.vbuffer.VBuffer method)@\spxentry{get\_pixel()}\spxextra{dpav.vbuffer.VBuffer method}}

\begin{fulllineitems}
\phantomsection\label{\detokenize{dpav:dpav.vbuffer.VBuffer.get_pixel}}
\pysigstartsignatures
\pysiglinewithargsret{\sphinxbfcode{\sphinxupquote{get\_pixel}}}{\emph{\DUrole{n}{coords}\DUrole{p}{:}\DUrole{w}{  }\DUrole{n}{list\DUrole{w}{  }\DUrole{p}{|}\DUrole{w}{  }tuple}}}{{ $\rightarrow$ int}}
\pysigstopsignatures
\sphinxAtStartPar
Return color value of chosen pixel.
\begin{quote}\begin{description}
\item[{Parameters}] \leavevmode
\sphinxAtStartPar
\sphinxstyleliteralstrong{\sphinxupquote{coords}} (\sphinxstyleliteralemphasis{\sphinxupquote{2\sphinxhyphen{}tuple}}\sphinxstyleliteralemphasis{\sphinxupquote{ or }}\sphinxstyleliteralemphasis{\sphinxupquote{list containing first and second index of pixel}}) \textendash{} 

\end{description}\end{quote}

\end{fulllineitems}

\index{load\_buffer\_from\_file() (dpav.vbuffer.VBuffer method)@\spxentry{load\_buffer\_from\_file()}\spxextra{dpav.vbuffer.VBuffer method}}

\begin{fulllineitems}
\phantomsection\label{\detokenize{dpav:dpav.vbuffer.VBuffer.load_buffer_from_file}}
\pysigstartsignatures
\pysiglinewithargsret{\sphinxbfcode{\sphinxupquote{load\_buffer\_from\_file}}}{\emph{\DUrole{n}{filename}\DUrole{p}{:}\DUrole{w}{  }\DUrole{n}{str}}}{{ $\rightarrow$ None}}
\pysigstopsignatures
\sphinxAtStartPar
Load binary file storing buffer contents, and write it to buffer.
\begin{quote}\begin{description}
\item[{Parameters}] \leavevmode
\sphinxAtStartPar
\sphinxstyleliteralstrong{\sphinxupquote{filename}} (\sphinxstyleliteralemphasis{\sphinxupquote{Path to a binary file containing numpy array data}}) \textendash{} 

\end{description}\end{quote}

\end{fulllineitems}

\index{save\_buffer\_to\_file() (dpav.vbuffer.VBuffer method)@\spxentry{save\_buffer\_to\_file()}\spxextra{dpav.vbuffer.VBuffer method}}

\begin{fulllineitems}
\phantomsection\label{\detokenize{dpav:dpav.vbuffer.VBuffer.save_buffer_to_file}}
\pysigstartsignatures
\pysiglinewithargsret{\sphinxbfcode{\sphinxupquote{save\_buffer\_to\_file}}}{\emph{\DUrole{n}{filename}\DUrole{p}{:}\DUrole{w}{  }\DUrole{n}{str}}}{{ $\rightarrow$ None}}
\pysigstopsignatures
\sphinxAtStartPar
Save contents of buffer to a binary file.
\begin{quote}\begin{description}
\item[{Parameters}] \leavevmode
\sphinxAtStartPar
\sphinxstyleliteralstrong{\sphinxupquote{filename}} (\sphinxstyleliteralemphasis{\sphinxupquote{The path and name of the file to write to}}) \textendash{} 

\end{description}\end{quote}

\end{fulllineitems}

\index{set\_buffer() (dpav.vbuffer.VBuffer method)@\spxentry{set\_buffer()}\spxextra{dpav.vbuffer.VBuffer method}}

\begin{fulllineitems}
\phantomsection\label{\detokenize{dpav:dpav.vbuffer.VBuffer.set_buffer}}
\pysigstartsignatures
\pysiglinewithargsret{\sphinxbfcode{\sphinxupquote{set\_buffer}}}{\emph{\DUrole{n}{buf}\DUrole{p}{:}\DUrole{w}{  }\DUrole{n}{numpy.ndarray}}}{{ $\rightarrow$ None}}
\pysigstopsignatures
\sphinxAtStartPar
Set the visual buffer to equal a provided 2D array of pixels.
\begin{quote}\begin{description}
\item[{Parameters}] \leavevmode
\sphinxAtStartPar
\sphinxstyleliteralstrong{\sphinxupquote{buf}} (\sphinxstyleliteralemphasis{\sphinxupquote{A 2\sphinxhyphen{}dimensional numpy array of integer color values}}) \textendash{} 

\end{description}\end{quote}

\end{fulllineitems}

\index{write\_pixel() (dpav.vbuffer.VBuffer method)@\spxentry{write\_pixel()}\spxextra{dpav.vbuffer.VBuffer method}}

\begin{fulllineitems}
\phantomsection\label{\detokenize{dpav:dpav.vbuffer.VBuffer.write_pixel}}
\pysigstartsignatures
\pysiglinewithargsret{\sphinxbfcode{\sphinxupquote{write\_pixel}}}{\emph{\DUrole{n}{coords}\DUrole{p}{:}\DUrole{w}{  }\DUrole{n}{list\DUrole{w}{  }\DUrole{p}{|}\DUrole{w}{  }tuple}}, \emph{\DUrole{n}{val}\DUrole{p}{:}\DUrole{w}{  }\DUrole{n}{int}}}{{ $\rightarrow$ None}}
\pysigstopsignatures
\sphinxAtStartPar
Sets pixel at specified coordinates to specified color.

\sphinxAtStartPar
Sets pixel at coordinates coords in buffer to hex value val
\begin{quote}\begin{description}
\item[{Parameters}] \leavevmode\begin{itemize}
\item {} 
\sphinxAtStartPar
\sphinxstyleliteralstrong{\sphinxupquote{coords}} (\sphinxstyleliteralemphasis{\sphinxupquote{Pixel coordinates}}\sphinxstyleliteralemphasis{\sphinxupquote{ (}}\sphinxstyleliteralemphasis{\sphinxupquote{an X and a Y}}\sphinxstyleliteralemphasis{\sphinxupquote{)}}) \textendash{} 

\item {} 
\sphinxAtStartPar
\sphinxstyleliteralstrong{\sphinxupquote{val}} (\sphinxstyleliteralemphasis{\sphinxupquote{The hex value of the desired color to change the pixel with}}) \textendash{} 

\end{itemize}

\end{description}\end{quote}

\sphinxAtStartPar
:raises TypeError : val is not type(int):
:raises ValueError : val is negative or greater than max color value (0xFFFFFF):

\end{fulllineitems}


\end{fulllineitems}



\subsection{dpav.window module}
\label{\detokenize{dpav:module-dpav.window}}\label{\detokenize{dpav:dpav-window-module}}\index{module@\spxentry{module}!dpav.window@\spxentry{dpav.window}}\index{dpav.window@\spxentry{dpav.window}!module@\spxentry{module}}\index{Window (class in dpav.window)@\spxentry{Window}\spxextra{class in dpav.window}}

\begin{fulllineitems}
\phantomsection\label{\detokenize{dpav:dpav.window.Window}}
\pysigstartsignatures
\pysiglinewithargsret{\sphinxbfcode{\sphinxupquote{class\DUrole{w}{  }}}\sphinxcode{\sphinxupquote{dpav.window.}}\sphinxbfcode{\sphinxupquote{Window}}}{\emph{\DUrole{n}{arg1}\DUrole{p}{:}\DUrole{w}{  }\DUrole{n}{Optional\DUrole{p}{{[}}{\hyperref[\detokenize{dpav:dpav.vbuffer.VBuffer}]{\sphinxcrossref{dpav.vbuffer.VBuffer}}}\DUrole{p}{{]}}}\DUrole{w}{  }\DUrole{o}{=}\DUrole{w}{  }\DUrole{default_value}{None}}, \emph{\DUrole{n}{scale}\DUrole{p}{:}\DUrole{w}{  }\DUrole{n}{float}\DUrole{w}{  }\DUrole{o}{=}\DUrole{w}{  }\DUrole{default_value}{1.0}}}{}
\pysigstopsignatures
\sphinxAtStartPar
Bases: \sphinxcode{\sphinxupquote{object}}

\sphinxAtStartPar
Handles Window capabilites of Python Direct Platform
Functions:
\begin{quote}
\begin{description}
\item[{Constructor:}] \leavevmode
\sphinxAtStartPar
\_\_init\_\_()

\item[{Setters:}] \leavevmode
\sphinxAtStartPar
set\_scale(int/float)
set\_vbuffer(VBuffer/np.ndarray,optional:int)

\item[{Getters:}] \leavevmode
\sphinxAtStartPar
get\_mouse\_pos()

\item[{Misc Methods:}] \leavevmode
\sphinxAtStartPar
open()
is\_open()
close()
update()

\item[{Private Methods:}] \leavevmode
\sphinxAtStartPar
\_update\_events(pygame.event)
\_build\_events\_dict()
\_write\_to\_screen()

\end{description}
\end{quote}
\index{Public (dpav.window.Window attribute)@\spxentry{Public}\spxextra{dpav.window.Window attribute}}

\begin{fulllineitems}
\phantomsection\label{\detokenize{dpav:dpav.window.Window.Public}}
\pysigstartsignatures
\pysigline{\sphinxbfcode{\sphinxupquote{Public}}}
\pysigstopsignatures
\sphinxAtStartPar
vbuffer:     active VBuffer object
scale:       number that scales up/down the size of the screen
\begin{quote}

\sphinxAtStartPar
(1.0 is unscaled)
\end{quote}
\begin{description}
\item[{events:      dictionary of string:bool event pairs,}] \leavevmode\begin{description}
\item[{example:}] \leavevmode
\sphinxAtStartPar
“l\_shift”: True  \textendash{} left shift is pressed down
“l\_shift”: False \textendash{} left shift is not pressed

\end{description}

\item[{eventq:      list of active events that occured since last update cycle}] \leavevmode\begin{description}
\item[{example:}] \leavevmode
\sphinxAtStartPar
{[}‘l\_shift’, ‘mouse’, ‘a’, ‘q’{]}

\end{description}

\end{description}

\sphinxAtStartPar
debug\_flag:  boolean flag if window object should output debug info to log
open\_flag:   boolean flag for if the window is active

\end{fulllineitems}

\index{Private (dpav.window.Window attribute)@\spxentry{Private}\spxextra{dpav.window.Window attribute}}

\begin{fulllineitems}
\phantomsection\label{\detokenize{dpav:dpav.window.Window.Private}}
\pysigstartsignatures
\pysigline{\sphinxbfcode{\sphinxupquote{Private}}}
\pysigstopsignatures\begin{description}
\item[{\_keydict:    int:string PyGame event mapping. PyGame events identifiers are}] \leavevmode
\sphinxAtStartPar
stored as ints. This attribute is used by the public events
variable to map from PyGame’s integer:boolean pairs to
our string:boolean pairs

\item[{\_surfaces:   Two PyGame Surfaces for swapping to reflect vbuffer changes and}] \leavevmode
\sphinxAtStartPar
enable in\sphinxhyphen{}place nparray modification

\end{description}

\sphinxAtStartPar
\_screen:     PyGame.display object, used for viewing vbuffer attribute

\end{fulllineitems}

\index{close() (dpav.window.Window method)@\spxentry{close()}\spxextra{dpav.window.Window method}}

\begin{fulllineitems}
\phantomsection\label{\detokenize{dpav:dpav.window.Window.close}}
\pysigstartsignatures
\pysiglinewithargsret{\sphinxbfcode{\sphinxupquote{close}}}{}{{ $\rightarrow$ None}}
\pysigstopsignatures
\sphinxAtStartPar
Closes the active instance of a pygame window
\begin{quote}\begin{description}
\item[{Raises}] \leavevmode
\sphinxAtStartPar
\sphinxstyleliteralstrong{\sphinxupquote{RuntimeError}} \textendash{} no active pygame window instances exists

\end{description}\end{quote}

\end{fulllineitems}

\index{get\_mouse\_pos() (dpav.window.Window method)@\spxentry{get\_mouse\_pos()}\spxextra{dpav.window.Window method}}

\begin{fulllineitems}
\phantomsection\label{\detokenize{dpav:dpav.window.Window.get_mouse_pos}}
\pysigstartsignatures
\pysiglinewithargsret{\sphinxbfcode{\sphinxupquote{get\_mouse\_pos}}}{\emph{\DUrole{n}{) \sphinxhyphen{}\textgreater{} (\textless{}class \textquotesingle{}int\textquotesingle{}\textgreater{}}}, \emph{\DUrole{n}{\textless{}class \textquotesingle{}int\textquotesingle{}\textgreater{}}}}{}
\pysigstopsignatures
\sphinxAtStartPar
Returns the current mouse location with respect to the pygame window instance
\begin{quote}\begin{description}
\item[{Raises}] \leavevmode
\sphinxAtStartPar
\sphinxstyleliteralstrong{\sphinxupquote{Runtime Error}} \textendash{} no active pygame window instances exists

\end{description}\end{quote}

\end{fulllineitems}

\index{is\_open() (dpav.window.Window method)@\spxentry{is\_open()}\spxextra{dpav.window.Window method}}

\begin{fulllineitems}
\phantomsection\label{\detokenize{dpav:dpav.window.Window.is_open}}
\pysigstartsignatures
\pysiglinewithargsret{\sphinxbfcode{\sphinxupquote{is\_open}}}{}{{ $\rightarrow$ bool}}
\pysigstopsignatures
\sphinxAtStartPar
Updates events on every call, used to abstract out PyGame
display calls and event loop
\subsubsection*{Example}
\begin{description}
\item[{if window.is\_open():}] \leavevmode
\sphinxAtStartPar
\# your code here

\end{description}
\begin{quote}\begin{description}
\item[{Returns}] \leavevmode
\sphinxAtStartPar
boolean denoting if the window is currently open

\end{description}\end{quote}

\end{fulllineitems}

\index{open() (dpav.window.Window method)@\spxentry{open()}\spxextra{dpav.window.Window method}}

\begin{fulllineitems}
\phantomsection\label{\detokenize{dpav:dpav.window.Window.open}}
\pysigstartsignatures
\pysiglinewithargsret{\sphinxbfcode{\sphinxupquote{open}}}{}{{ $\rightarrow$ None}}
\pysigstopsignatures
\sphinxAtStartPar
Creates and runs pygame window in a new thread

\end{fulllineitems}

\index{set\_scale() (dpav.window.Window method)@\spxentry{set\_scale()}\spxextra{dpav.window.Window method}}

\begin{fulllineitems}
\phantomsection\label{\detokenize{dpav:dpav.window.Window.set_scale}}
\pysigstartsignatures
\pysiglinewithargsret{\sphinxbfcode{\sphinxupquote{set\_scale}}}{\emph{\DUrole{n}{scale}\DUrole{p}{:}\DUrole{w}{  }\DUrole{n}{float}}}{{ $\rightarrow$ None}}
\pysigstopsignatures
\sphinxAtStartPar
Sets the window scale

\end{fulllineitems}

\index{set\_vbuffer() (dpav.window.Window method)@\spxentry{set\_vbuffer()}\spxextra{dpav.window.Window method}}

\begin{fulllineitems}
\phantomsection\label{\detokenize{dpav:dpav.window.Window.set_vbuffer}}
\pysigstartsignatures
\pysiglinewithargsret{\sphinxbfcode{\sphinxupquote{set\_vbuffer}}}{\emph{\DUrole{n}{arg1}\DUrole{p}{:}\DUrole{w}{  }\DUrole{n}{{\hyperref[\detokenize{dpav:dpav.vbuffer.VBuffer}]{\sphinxcrossref{dpav.vbuffer.VBuffer}}}}}}{{ $\rightarrow$ None}}
\pysigstopsignatures
\sphinxAtStartPar
Sets the vbuffer/nparray object to display on screen
\begin{quote}\begin{description}
\item[{Parameters}] \leavevmode
\sphinxAtStartPar
\sphinxstyleliteralstrong{\sphinxupquote{arg1}} \textendash{} VBuffer/np.ndarray

\item[{Raises}] \leavevmode\begin{itemize}
\item {} 
\sphinxAtStartPar
\sphinxstyleliteralstrong{\sphinxupquote{TypeError}} \textendash{} arg1 VBuffer/np.ndarray type check

\item {} 
\sphinxAtStartPar
\sphinxstyleliteralstrong{\sphinxupquote{TypeError}} \textendash{} scale int/float type check

\end{itemize}

\end{description}\end{quote}

\end{fulllineitems}

\index{update() (dpav.window.Window method)@\spxentry{update()}\spxextra{dpav.window.Window method}}

\begin{fulllineitems}
\phantomsection\label{\detokenize{dpav:dpav.window.Window.update}}
\pysigstartsignatures
\pysiglinewithargsret{\sphinxbfcode{\sphinxupquote{update}}}{}{{ $\rightarrow$ None}}
\pysigstopsignatures
\sphinxAtStartPar
Pygame event abstraction, called at end of pygame loop.
Optional function if is\_open() is used
\begin{quote}\begin{description}
\item[{Raises}] \leavevmode
\sphinxAtStartPar
\sphinxstyleliteralstrong{\sphinxupquote{Runtime Error}} \textendash{} No active pygame window

\end{description}\end{quote}

\end{fulllineitems}


\end{fulllineitems}



\subsection{Module contents}
\label{\detokenize{dpav:module-dpav}}\label{\detokenize{dpav:module-contents}}\index{module@\spxentry{module}!dpav@\spxentry{dpav}}\index{dpav@\spxentry{dpav}!module@\spxentry{module}}\index{Audio (class in dpav)@\spxentry{Audio}\spxextra{class in dpav}}

\begin{fulllineitems}
\phantomsection\label{\detokenize{dpav:dpav.Audio}}
\pysigstartsignatures
\pysigline{\sphinxbfcode{\sphinxupquote{class\DUrole{w}{  }}}\sphinxcode{\sphinxupquote{dpav.}}\sphinxbfcode{\sphinxupquote{Audio}}}
\pysigstopsignatures
\sphinxAtStartPar
Bases: \sphinxcode{\sphinxupquote{object}}

\sphinxAtStartPar
Handles Audio capabilities of Python Direct Platform.
\begin{description}
\item[{Functions:}] \leavevmode\begin{description}
\item[{Constructor:}] \leavevmode
\sphinxAtStartPar
\_\_init\_\_()

\item[{Functions:}] \leavevmode\begin{description}
\item[{play\_sound(Hz, length)}] \leavevmode\begin{description}
\item[{If audio buffer is set:}] \leavevmode
\sphinxAtStartPar
play\_sound()

\end{description}

\end{description}

\sphinxAtStartPar
play\_sample(string\_name\_of\_wav\_file)

\item[{Setters:}] \leavevmode
\sphinxAtStartPar
set\_audio\_buffer(numpyarray)
set\_audio\_device(int)
set\_waveform(waveform)

\item[{Getters:}] \leavevmode
\sphinxAtStartPar
get\_bit\_number()\sphinxhyphen{}\textgreater{}int
get\_sample\_rate()\sphinxhyphen{}\textgreater{}int
get\_audio\_buffer()
get\_audio\_device()\sphinxhyphen{}\textgreater{}Returns int corresponding to audio device

\item[{Misc:}] \leavevmode
\sphinxAtStartPar
list\_audio\_devices()
wait\_for\_sound\_end()

\end{description}

\end{description}
\index{get\_audio\_buffer() (dpav.Audio method)@\spxentry{get\_audio\_buffer()}\spxextra{dpav.Audio method}}

\begin{fulllineitems}
\phantomsection\label{\detokenize{dpav:dpav.Audio.get_audio_buffer}}
\pysigstartsignatures
\pysiglinewithargsret{\sphinxbfcode{\sphinxupquote{get\_audio\_buffer}}}{}{}
\pysigstopsignatures
\sphinxAtStartPar
Returns the audio buffer of the Audio class
\begin{description}
\item[{Description:}] \leavevmode
\sphinxAtStartPar
This will return none if the audio buffer has not been set by the set\_audio\_buffer method.

\sphinxAtStartPar
audioobject.get\_audio\_buffer()

\end{description}
\begin{quote}\begin{description}
\item[{Parameters}] \leavevmode
\sphinxAtStartPar
\sphinxstyleliteralstrong{\sphinxupquote{None}} \textendash{} 

\item[{Returns}] \leavevmode
\sphinxAtStartPar
numpy array

\item[{Return type}] \leavevmode
\sphinxAtStartPar
self.\_audio\_buffer

\end{description}\end{quote}

\end{fulllineitems}

\index{get\_audio\_device() (dpav.Audio method)@\spxentry{get\_audio\_device()}\spxextra{dpav.Audio method}}

\begin{fulllineitems}
\phantomsection\label{\detokenize{dpav:dpav.Audio.get_audio_device}}
\pysigstartsignatures
\pysiglinewithargsret{\sphinxbfcode{\sphinxupquote{get\_audio\_device}}}{}{{ $\rightarrow$ int}}
\pysigstopsignatures
\sphinxAtStartPar
Gets the current audio device number of the Audio Class
\begin{description}
\item[{Description:}] \leavevmode
\sphinxAtStartPar
Assuming audioobject.set\_audio\_device(2) is called,
audioobject.get\_audio\_device() would return 2 {[}index of audio device in audioobject.list\_audio\_devices(){]}

\end{description}
\begin{quote}\begin{description}
\item[{Parameters}] \leavevmode
\sphinxAtStartPar
\sphinxstyleliteralstrong{\sphinxupquote{None}} \textendash{} 

\end{description}\end{quote}
\begin{description}
\item[{Returns}] \leavevmode
\sphinxAtStartPar
self.\_audio\_device: int value

\end{description}
\subsubsection*{Notes}

\sphinxAtStartPar
Returns the integer value of the device not the device name

\end{fulllineitems}

\index{get\_bit\_number() (dpav.Audio method)@\spxentry{get\_bit\_number()}\spxextra{dpav.Audio method}}

\begin{fulllineitems}
\phantomsection\label{\detokenize{dpav:dpav.Audio.get_bit_number}}
\pysigstartsignatures
\pysiglinewithargsret{\sphinxbfcode{\sphinxupquote{get\_bit\_number}}}{}{{ $\rightarrow$ int}}
\pysigstopsignatures
\sphinxAtStartPar
Gets the bit rate of the Audio class
\begin{description}
\item[{Description:}] \leavevmode
\sphinxAtStartPar
Bit rate currently locked to 16 bits

\end{description}
\begin{quote}\begin{description}
\item[{Parameters}] \leavevmode
\sphinxAtStartPar
\sphinxstyleliteralstrong{\sphinxupquote{None}} \textendash{} 

\item[{Returns}] \leavevmode
\sphinxAtStartPar
The bit rate of the Audio class \sphinxhyphen{} int value

\item[{Return type}] \leavevmode
\sphinxAtStartPar
self.\_bit\_number

\end{description}\end{quote}

\end{fulllineitems}

\index{get\_sample\_rate() (dpav.Audio method)@\spxentry{get\_sample\_rate()}\spxextra{dpav.Audio method}}

\begin{fulllineitems}
\phantomsection\label{\detokenize{dpav:dpav.Audio.get_sample_rate}}
\pysigstartsignatures
\pysiglinewithargsret{\sphinxbfcode{\sphinxupquote{get\_sample\_rate}}}{}{{ $\rightarrow$ int}}
\pysigstopsignatures
\sphinxAtStartPar
Gets the sample rate of the Audio class.
\begin{description}
\item[{Description:}] \leavevmode
\sphinxAtStartPar
Sample rate is currently locked to 44100

\end{description}
\begin{quote}\begin{description}
\item[{Parameters}] \leavevmode
\sphinxAtStartPar
\sphinxstyleliteralstrong{\sphinxupquote{None}} \textendash{} 

\item[{Returns}] \leavevmode
\sphinxAtStartPar
The sample rate of the audioClass \sphinxhyphen{} int value

\item[{Return type}] \leavevmode
\sphinxAtStartPar
self.\_sample\_rate

\end{description}\end{quote}

\end{fulllineitems}

\index{list\_audio\_devices() (dpav.Audio method)@\spxentry{list\_audio\_devices()}\spxextra{dpav.Audio method}}

\begin{fulllineitems}
\phantomsection\label{\detokenize{dpav:dpav.Audio.list_audio_devices}}
\pysigstartsignatures
\pysiglinewithargsret{\sphinxbfcode{\sphinxupquote{list\_audio\_devices}}}{}{{ $\rightarrow$ None}}
\pysigstopsignatures
\sphinxAtStartPar
Lists the output devices on your system and adds to list self.\_devices
\begin{description}
\item[{Description:}] \leavevmode
\sphinxAtStartPar
Run this function before using set\_audio\_device() to add devices to the list devices

\sphinxAtStartPar
audioobject.list\_audio\_devices()
0 Speakers (Realtek(R) Audio)
1 VGA248 (2\sphinxhyphen{}NVIDIA High Def Audio)
2 Speakers (HyperX Cloud II Wireless)

\end{description}
\begin{quote}\begin{description}
\item[{Parameters}] \leavevmode
\sphinxAtStartPar
\sphinxstyleliteralstrong{\sphinxupquote{None}} \textendash{} 

\item[{Returns}] \leavevmode
\sphinxAtStartPar
None

\end{description}\end{quote}

\end{fulllineitems}

\index{play\_sample() (dpav.Audio method)@\spxentry{play\_sample()}\spxextra{dpav.Audio method}}

\begin{fulllineitems}
\phantomsection\label{\detokenize{dpav:dpav.Audio.play_sample}}
\pysigstartsignatures
\pysiglinewithargsret{\sphinxbfcode{\sphinxupquote{play\_sample}}}{\emph{\DUrole{n}{sample\_name}\DUrole{p}{:}\DUrole{w}{  }\DUrole{n}{str}}}{{ $\rightarrow$ None}}
\pysigstopsignatures
\sphinxAtStartPar
Plays sounds that are wav, ogg or mp3 files.
\begin{description}
\item[{Description:}] \leavevmode
\sphinxAtStartPar
audioobject.play\_sample(mypath.mp3) would play sounds from the file mypath.mp3

\end{description}
\begin{quote}\begin{description}
\item[{Parameters}] \leavevmode
\sphinxAtStartPar
\sphinxstyleliteralstrong{\sphinxupquote{sample\_name}} \textendash{} String path or name of sound

\item[{Returns}] \leavevmode
\sphinxAtStartPar
None

\end{description}\end{quote}

\end{fulllineitems}

\index{play\_sound() (dpav.Audio method)@\spxentry{play\_sound()}\spxextra{dpav.Audio method}}

\begin{fulllineitems}
\phantomsection\label{\detokenize{dpav:dpav.Audio.play_sound}}
\pysigstartsignatures
\pysiglinewithargsret{\sphinxbfcode{\sphinxupquote{play\_sound}}}{\emph{\DUrole{n}{input\_frequency}\DUrole{o}{=}\DUrole{default_value}{0}}, \emph{\DUrole{n}{input\_duration}\DUrole{o}{=}\DUrole{default_value}{0}}}{{ $\rightarrow$ None}}
\pysigstopsignatures
\sphinxAtStartPar
Primary sound playing method of the audio class.
\begin{description}
\item[{Description:}] \leavevmode
\sphinxAtStartPar
Play sounds directly from this function
Need to run set\_audio\_device() or will default to the default audio device
You can use set\_waveform to change the type.
play\_sound is somewhat overloaded to where if you have an audioBuffer set using set\_audio\_buffer, you can call play\_sound()
\begin{quote}

\sphinxAtStartPar
and it will play whatever that audio\_buffer is e.g. wav files
Example in examples/custombuffer.py
\end{quote}

\sphinxAtStartPar
play\_sound(440, 1) would play an A note for one second with the sin waveform set.

\end{description}
\begin{quote}\begin{description}
\item[{Parameters}] \leavevmode\begin{itemize}
\item {} 
\sphinxAtStartPar
\sphinxstyleliteralstrong{\sphinxupquote{input\_frequency}} \textendash{} int value \sphinxhyphen{} input frequency in Hz

\item {} 
\sphinxAtStartPar
\sphinxstyleliteralstrong{\sphinxupquote{input\_duration}} \textendash{} int value \sphinxhyphen{} duration in seconds

\end{itemize}

\item[{Raises}] \leavevmode
\sphinxAtStartPar
\sphinxstyleliteralstrong{\sphinxupquote{TypeError}} \textendash{} If input\_duration not a number, or \textless{} 0

\item[{Returns}] \leavevmode
\sphinxAtStartPar
None

\end{description}\end{quote}

\end{fulllineitems}

\index{set\_audio\_buffer() (dpav.Audio method)@\spxentry{set\_audio\_buffer()}\spxextra{dpav.Audio method}}

\begin{fulllineitems}
\phantomsection\label{\detokenize{dpav:dpav.Audio.set_audio_buffer}}
\pysigstartsignatures
\pysiglinewithargsret{\sphinxbfcode{\sphinxupquote{set\_audio\_buffer}}}{\emph{\DUrole{n}{ab}}}{{ $\rightarrow$ None}}
\pysigstopsignatures
\sphinxAtStartPar
Sets the audio buffer of the Audio Class.
\begin{description}
\item[{Description:}] \leavevmode
\sphinxAtStartPar
The audio buffer needs to have two rows so that way stereo works as intended.
You can set the audio buffer to wav file data by fetching numpy arrays using wav or scipy,
however only 16 bit waves are supported. This process can be seen in custom\_buffer.py w/ the
utility function sixteenWavtoRawData
\begin{description}
\item[{Examples:}] \leavevmode
\sphinxAtStartPar
\# 44100 = sample rate
\# 32767 is 2 \textasciicircum{} (our bit depth \sphinxhyphen{}1)\sphinxhyphen{}1 and is essentially the number of samples per time stamp
\# 260 and 290 are our tones in hz
\# Below generates a buffer 1 second long of sin wave data\sphinxhyphen{}identical to the method used in house
data = numpy.zeros((44100, 2), dtype=numpy.int16)
for s in range(44100):
\begin{quote}

\sphinxAtStartPar
t = float(s) / 44100
data{[}s{]}{[}0{]} = int(round(32767 * math.sin(2 * math.pi * 260 * t)))
data{[}s{]}{[}1{]} = int(round(32767 * math.sin(2 * math.pi * 290 * t)))
\end{quote}

\sphinxAtStartPar
audioobject.set\_audio\_buffer(data)

\end{description}

\end{description}
\begin{quote}\begin{description}
\item[{Parameters}] \leavevmode
\sphinxAtStartPar
\sphinxstyleliteralstrong{\sphinxupquote{ab}} \textendash{} numpy array of shape(samples, channels) e.g. ab{[}44100{]}{[}2{]}

\item[{Returns}] \leavevmode
\sphinxAtStartPar
None

\end{description}\end{quote}

\end{fulllineitems}

\index{set\_audio\_device() (dpav.Audio method)@\spxentry{set\_audio\_device()}\spxextra{dpav.Audio method}}

\begin{fulllineitems}
\phantomsection\label{\detokenize{dpav:dpav.Audio.set_audio_device}}
\pysigstartsignatures
\pysiglinewithargsret{\sphinxbfcode{\sphinxupquote{set\_audio\_device}}}{\emph{\DUrole{n}{device}\DUrole{p}{:}\DUrole{w}{  }\DUrole{n}{int}}}{{ $\rightarrow$ int}}
\pysigstopsignatures
\sphinxAtStartPar
Sets the current audio device of the Audio class.
\begin{description}
\item[{Description:}] \leavevmode
\sphinxAtStartPar
This can only be set ONCE per instance. To change devices, del the current instance
set the new device, and continue
This needs to be run after list\_audio\_device() in order to see list of audio devices
If not run the device will default to the current device being used by the machine

\sphinxAtStartPar
audioobject.set\_audio\_device(2)
Based on example in list\_audio\_devices() this would change the device to Speakers (HyperX Cloud II Wireless)

\end{description}
\begin{quote}\begin{description}
\item[{Parameters}] \leavevmode
\sphinxAtStartPar
\sphinxstyleliteralstrong{\sphinxupquote{device}} \textendash{} int value \sphinxhyphen{} see all int values for each device by running list\_audio\_devices()

\item[{Returns}] \leavevmode
\sphinxAtStartPar
None

\end{description}\end{quote}

\end{fulllineitems}

\index{set\_waveform() (dpav.Audio method)@\spxentry{set\_waveform()}\spxextra{dpav.Audio method}}

\begin{fulllineitems}
\phantomsection\label{\detokenize{dpav:dpav.Audio.set_waveform}}
\pysigstartsignatures
\pysiglinewithargsret{\sphinxbfcode{\sphinxupquote{set\_waveform}}}{\emph{\DUrole{n}{wave}}}{{ $\rightarrow$ None}}
\pysigstopsignatures
\sphinxAtStartPar
Sets the expression governing the wave form playing
\begin{description}
\item[{Description:}] \leavevmode
\sphinxAtStartPar
play\_audio uses this in buffer generation

\sphinxAtStartPar
audioobject.set\_waveform(object.wave\_table.sin)
This would change to the waveform sin contained in the wave\_table class
The wave functions need to take in a input frequency as well as a timestep parameter
to solve for a particular frequency at a given time step. See wave\_table for an example of this.

\end{description}
\begin{quote}\begin{description}
\item[{Parameters}] \leavevmode
\sphinxAtStartPar
\sphinxstyleliteralstrong{\sphinxupquote{Wave}} \textendash{} takes a mathematical expression function ‘pointer’ in the form of f(inputfreq, timestep)

\item[{Returns}] \leavevmode
\sphinxAtStartPar
None

\end{description}\end{quote}

\end{fulllineitems}

\index{wait\_for\_sound\_end() (dpav.Audio method)@\spxentry{wait\_for\_sound\_end()}\spxextra{dpav.Audio method}}

\begin{fulllineitems}
\phantomsection\label{\detokenize{dpav:dpav.Audio.wait_for_sound_end}}
\pysigstartsignatures
\pysiglinewithargsret{\sphinxbfcode{\sphinxupquote{wait\_for\_sound\_end}}}{}{}
\pysigstopsignatures
\sphinxAtStartPar
Function call that is placed at the end of scripts without a pygame window instance so sounds play to their full duration without a
\begin{description}
\item[{Description:}] \leavevmode
\sphinxAtStartPar
Placed at the end of python files that do not have loops. Otherwise, sounds would be cut off prematurely.
\begin{description}
\item[{Example:}] \leavevmode
\sphinxAtStartPar
play\_sound(440, 10)
wait\_for\_sound\_end() \# This prevents the process from closing out before the sound ends.

\end{description}

\end{description}
\begin{quote}\begin{description}
\item[{Parameters}] \leavevmode
\sphinxAtStartPar
\sphinxstyleliteralstrong{\sphinxupquote{None}} \textendash{} 

\item[{Returns}] \leavevmode
\sphinxAtStartPar
None

\end{description}\end{quote}

\sphinxAtStartPar
Notes:

\end{fulllineitems}


\end{fulllineitems}

\index{VBuffer (class in dpav)@\spxentry{VBuffer}\spxextra{class in dpav}}

\begin{fulllineitems}
\phantomsection\label{\detokenize{dpav:dpav.VBuffer}}
\pysigstartsignatures
\pysiglinewithargsret{\sphinxbfcode{\sphinxupquote{class\DUrole{w}{  }}}\sphinxcode{\sphinxupquote{dpav.}}\sphinxbfcode{\sphinxupquote{VBuffer}}}{\emph{\DUrole{n}{arg1}\DUrole{p}{:}\DUrole{w}{  }\DUrole{n}{list\DUrole{w}{  }\DUrole{p}{|}\DUrole{w}{  }tuple\DUrole{w}{  }\DUrole{p}{|}\DUrole{w}{  }numpy.ndarray}\DUrole{w}{  }\DUrole{o}{=}\DUrole{w}{  }\DUrole{default_value}{(800, 600)}}}{}
\pysigstopsignatures
\sphinxAtStartPar
Bases: \sphinxcode{\sphinxupquote{object}}

\sphinxAtStartPar
Visual buffer for the Python Direct Platform

\sphinxAtStartPar
Holds a 2D array of hex color values. Each element represents a pixel,
whose coordinates are its index. VBuffer can be loaded and displayed by
the window class.
\begin{quote}\begin{description}
\item[{Parameters}] \leavevmode
\sphinxAtStartPar
\sphinxstyleliteralstrong{\sphinxupquote{arg1}} (\sphinxstyleliteralemphasis{\sphinxupquote{\{}}\sphinxstyleliteralemphasis{\sphinxupquote{(}}\sphinxstyleliteralemphasis{\sphinxupquote{int}}\sphinxstyleliteralemphasis{\sphinxupquote{, }}\sphinxstyleliteralemphasis{\sphinxupquote{int}}\sphinxstyleliteralemphasis{\sphinxupquote{)}}\sphinxstyleliteralemphasis{\sphinxupquote{|}}\sphinxstyleliteralemphasis{\sphinxupquote{np.ndarray}}\sphinxstyleliteralemphasis{\sphinxupquote{(}}\sphinxstyleliteralemphasis{\sphinxupquote{int}}\sphinxstyleliteralemphasis{\sphinxupquote{, }}\sphinxstyleliteralemphasis{\sphinxupquote{int}}\sphinxstyleliteralemphasis{\sphinxupquote{)}}\sphinxstyleliteralemphasis{\sphinxupquote{\}}}) \textendash{} 
\sphinxAtStartPar
Either array dimensions or a 2\sphinxhyphen{}dimensional numpy array of integers

\sphinxAtStartPar
If dimensions, will create zeroed\sphinxhyphen{}out 2D array of the selected
dimensions. Defaults to 800x600.

\sphinxAtStartPar
If numpy array, will set buffer to the contents of that array.


\end{description}\end{quote}


\begin{fulllineitems}

\pysigstartsignatures
\pysigline{\sphinxbfcode{\sphinxupquote{Constructor:}}}
\pysigstopsignatures
\sphinxAtStartPar
\_\_init\_\_(self, arg1=(800, 600)) \sphinxhyphen{}\textgreater{} None

\end{fulllineitems}



\begin{fulllineitems}

\pysigstartsignatures
\pysigline{\sphinxbfcode{\sphinxupquote{Overloads:}}}
\pysigstopsignatures
\sphinxAtStartPar
\_\_getitem\_\_(self, idx) \sphinxhyphen{}\textgreater{} int
\_\_setitem\_\_(self, idx, val) \sphinxhyphen{}\textgreater{} None
\_\_len\_\_(self) \sphinxhyphen{}\textgreater{} int

\end{fulllineitems}



\begin{fulllineitems}

\pysigstartsignatures
\pysigline{\sphinxbfcode{\sphinxupquote{properties:}}}
\pysigstopsignatures\begin{description}
\item[{getter:}] \leavevmode
\sphinxAtStartPar
dimensions(self) \sphinxhyphen{}\textgreater{} (int, int)

\item[{setter:}] \leavevmode
\sphinxAtStartPar
dimensions(self, val) \sphinxhyphen{}\textgreater{} None

\end{description}

\end{fulllineitems}



\begin{fulllineitems}

\pysigstartsignatures
\pysigline{\sphinxbfcode{\sphinxupquote{Setter:}}}
\pysigstopsignatures
\sphinxAtStartPar
write\_pixel(self, coords, val) \sphinxhyphen{}\textgreater{} None
set\_buffer(self, buf) \sphinxhyphen{}\textgreater{} None
clear(self) \sphinxhyphen{}\textgreater{} None
fill(self, color: int) \sphinxhyphen{}\textgreater{} None

\end{fulllineitems}



\begin{fulllineitems}

\pysigstartsignatures
\pysigline{\sphinxbfcode{\sphinxupquote{Getters:}}}
\pysigstopsignatures
\sphinxAtStartPar
get\_pixel(self, coords) \sphinxhyphen{}\textgreater{} int
get\_dimensions(self) \sphinxhyphen{}\textgreater{} (int, int)

\end{fulllineitems}



\begin{fulllineitems}

\pysigstartsignatures
\pysigline{\sphinxbfcode{\sphinxupquote{File~I/O:}}}
\pysigstopsignatures
\sphinxAtStartPar
save\_buffer\_to\_file(self, filename) \sphinxhyphen{}\textgreater{} None
load\_buffer\_from\_file(self, filename) \sphinxhyphen{}\textgreater{} None

\end{fulllineitems}



\begin{fulllineitems}

\pysigstartsignatures
\pysigline{\sphinxbfcode{\sphinxupquote{Error~Checking:}}}
\pysigstopsignatures
\sphinxAtStartPar
\_check\_numpy\_arr(self,arg1,arg\_name,method\_name) \sphinxhyphen{}\textgreater{} None
\_check\_coord\_type(self, coords, arg\_name, method\_name) \sphinxhyphen{}\textgreater{} None
\_check\_coord\_vals(self, x, y, method\_name) \sphinxhyphen{}\textgreater{} None

\end{fulllineitems}

\index{clear() (dpav.VBuffer method)@\spxentry{clear()}\spxextra{dpav.VBuffer method}}

\begin{fulllineitems}
\phantomsection\label{\detokenize{dpav:dpav.VBuffer.clear}}
\pysigstartsignatures
\pysiglinewithargsret{\sphinxbfcode{\sphinxupquote{clear}}}{}{{ $\rightarrow$ None}}
\pysigstopsignatures
\sphinxAtStartPar
Set every pixel in buffer to 0 (hex value for black).

\end{fulllineitems}

\index{dimensions (dpav.VBuffer property)@\spxentry{dimensions}\spxextra{dpav.VBuffer property}}

\begin{fulllineitems}
\phantomsection\label{\detokenize{dpav:dpav.VBuffer.dimensions}}
\pysigstartsignatures
\pysigline{\sphinxbfcode{\sphinxupquote{property\DUrole{w}{  }}}\sphinxbfcode{\sphinxupquote{dimensions}}\sphinxbfcode{\sphinxupquote{\DUrole{p}{:}\DUrole{w}{  }list\DUrole{w}{  }\DUrole{p}{|}\DUrole{w}{  }tuple}}}
\pysigstopsignatures
\sphinxAtStartPar
Return dimensions of buffer.

\end{fulllineitems}

\index{fill() (dpav.VBuffer method)@\spxentry{fill()}\spxextra{dpav.VBuffer method}}

\begin{fulllineitems}
\phantomsection\label{\detokenize{dpav:dpav.VBuffer.fill}}
\pysigstartsignatures
\pysiglinewithargsret{\sphinxbfcode{\sphinxupquote{fill}}}{\emph{\DUrole{n}{color}\DUrole{p}{:}\DUrole{w}{  }\DUrole{n}{int}}}{{ $\rightarrow$ None}}
\pysigstopsignatures
\sphinxAtStartPar
Set every pixel in the buffer to a given color.
\begin{quote}\begin{description}
\item[{Parameters}] \leavevmode
\sphinxAtStartPar
\sphinxstyleliteralstrong{\sphinxupquote{color}} (\sphinxstyleliteralemphasis{\sphinxupquote{Hex color code}}) \textendash{} 

\end{description}\end{quote}

\end{fulllineitems}

\index{get\_dimensions() (dpav.VBuffer method)@\spxentry{get\_dimensions()}\spxextra{dpav.VBuffer method}}

\begin{fulllineitems}
\phantomsection\label{\detokenize{dpav:dpav.VBuffer.get_dimensions}}
\pysigstartsignatures
\pysiglinewithargsret{\sphinxbfcode{\sphinxupquote{get\_dimensions}}}{}{{ $\rightarrow$ list\DUrole{w}{  }\DUrole{p}{|}\DUrole{w}{  }tuple}}
\pysigstopsignatures
\sphinxAtStartPar
Return dimensions of visual buffer array.

\end{fulllineitems}

\index{get\_pixel() (dpav.VBuffer method)@\spxentry{get\_pixel()}\spxextra{dpav.VBuffer method}}

\begin{fulllineitems}
\phantomsection\label{\detokenize{dpav:dpav.VBuffer.get_pixel}}
\pysigstartsignatures
\pysiglinewithargsret{\sphinxbfcode{\sphinxupquote{get\_pixel}}}{\emph{\DUrole{n}{coords}\DUrole{p}{:}\DUrole{w}{  }\DUrole{n}{list\DUrole{w}{  }\DUrole{p}{|}\DUrole{w}{  }tuple}}}{{ $\rightarrow$ int}}
\pysigstopsignatures
\sphinxAtStartPar
Return color value of chosen pixel.
\begin{quote}\begin{description}
\item[{Parameters}] \leavevmode
\sphinxAtStartPar
\sphinxstyleliteralstrong{\sphinxupquote{coords}} (\sphinxstyleliteralemphasis{\sphinxupquote{2\sphinxhyphen{}tuple}}\sphinxstyleliteralemphasis{\sphinxupquote{ or }}\sphinxstyleliteralemphasis{\sphinxupquote{list containing first and second index of pixel}}) \textendash{} 

\end{description}\end{quote}

\end{fulllineitems}

\index{load\_buffer\_from\_file() (dpav.VBuffer method)@\spxentry{load\_buffer\_from\_file()}\spxextra{dpav.VBuffer method}}

\begin{fulllineitems}
\phantomsection\label{\detokenize{dpav:dpav.VBuffer.load_buffer_from_file}}
\pysigstartsignatures
\pysiglinewithargsret{\sphinxbfcode{\sphinxupquote{load\_buffer\_from\_file}}}{\emph{\DUrole{n}{filename}\DUrole{p}{:}\DUrole{w}{  }\DUrole{n}{str}}}{{ $\rightarrow$ None}}
\pysigstopsignatures
\sphinxAtStartPar
Load binary file storing buffer contents, and write it to buffer.
\begin{quote}\begin{description}
\item[{Parameters}] \leavevmode
\sphinxAtStartPar
\sphinxstyleliteralstrong{\sphinxupquote{filename}} (\sphinxstyleliteralemphasis{\sphinxupquote{Path to a binary file containing numpy array data}}) \textendash{} 

\end{description}\end{quote}

\end{fulllineitems}

\index{save\_buffer\_to\_file() (dpav.VBuffer method)@\spxentry{save\_buffer\_to\_file()}\spxextra{dpav.VBuffer method}}

\begin{fulllineitems}
\phantomsection\label{\detokenize{dpav:dpav.VBuffer.save_buffer_to_file}}
\pysigstartsignatures
\pysiglinewithargsret{\sphinxbfcode{\sphinxupquote{save\_buffer\_to\_file}}}{\emph{\DUrole{n}{filename}\DUrole{p}{:}\DUrole{w}{  }\DUrole{n}{str}}}{{ $\rightarrow$ None}}
\pysigstopsignatures
\sphinxAtStartPar
Save contents of buffer to a binary file.
\begin{quote}\begin{description}
\item[{Parameters}] \leavevmode
\sphinxAtStartPar
\sphinxstyleliteralstrong{\sphinxupquote{filename}} (\sphinxstyleliteralemphasis{\sphinxupquote{The path and name of the file to write to}}) \textendash{} 

\end{description}\end{quote}

\end{fulllineitems}

\index{set\_buffer() (dpav.VBuffer method)@\spxentry{set\_buffer()}\spxextra{dpav.VBuffer method}}

\begin{fulllineitems}
\phantomsection\label{\detokenize{dpav:dpav.VBuffer.set_buffer}}
\pysigstartsignatures
\pysiglinewithargsret{\sphinxbfcode{\sphinxupquote{set\_buffer}}}{\emph{\DUrole{n}{buf}\DUrole{p}{:}\DUrole{w}{  }\DUrole{n}{numpy.ndarray}}}{{ $\rightarrow$ None}}
\pysigstopsignatures
\sphinxAtStartPar
Set the visual buffer to equal a provided 2D array of pixels.
\begin{quote}\begin{description}
\item[{Parameters}] \leavevmode
\sphinxAtStartPar
\sphinxstyleliteralstrong{\sphinxupquote{buf}} (\sphinxstyleliteralemphasis{\sphinxupquote{A 2\sphinxhyphen{}dimensional numpy array of integer color values}}) \textendash{} 

\end{description}\end{quote}

\end{fulllineitems}

\index{write\_pixel() (dpav.VBuffer method)@\spxentry{write\_pixel()}\spxextra{dpav.VBuffer method}}

\begin{fulllineitems}
\phantomsection\label{\detokenize{dpav:dpav.VBuffer.write_pixel}}
\pysigstartsignatures
\pysiglinewithargsret{\sphinxbfcode{\sphinxupquote{write\_pixel}}}{\emph{\DUrole{n}{coords}\DUrole{p}{:}\DUrole{w}{  }\DUrole{n}{list\DUrole{w}{  }\DUrole{p}{|}\DUrole{w}{  }tuple}}, \emph{\DUrole{n}{val}\DUrole{p}{:}\DUrole{w}{  }\DUrole{n}{int}}}{{ $\rightarrow$ None}}
\pysigstopsignatures
\sphinxAtStartPar
Sets pixel at specified coordinates to specified color.

\sphinxAtStartPar
Sets pixel at coordinates coords in buffer to hex value val
\begin{quote}\begin{description}
\item[{Parameters}] \leavevmode\begin{itemize}
\item {} 
\sphinxAtStartPar
\sphinxstyleliteralstrong{\sphinxupquote{coords}} (\sphinxstyleliteralemphasis{\sphinxupquote{Pixel coordinates}}\sphinxstyleliteralemphasis{\sphinxupquote{ (}}\sphinxstyleliteralemphasis{\sphinxupquote{an X and a Y}}\sphinxstyleliteralemphasis{\sphinxupquote{)}}) \textendash{} 

\item {} 
\sphinxAtStartPar
\sphinxstyleliteralstrong{\sphinxupquote{val}} (\sphinxstyleliteralemphasis{\sphinxupquote{The hex value of the desired color to change the pixel with}}) \textendash{} 

\end{itemize}

\end{description}\end{quote}

\sphinxAtStartPar
:raises TypeError : val is not type(int):
:raises ValueError : val is negative or greater than max color value (0xFFFFFF):

\end{fulllineitems}


\end{fulllineitems}

\index{Window (class in dpav)@\spxentry{Window}\spxextra{class in dpav}}

\begin{fulllineitems}
\phantomsection\label{\detokenize{dpav:dpav.Window}}
\pysigstartsignatures
\pysiglinewithargsret{\sphinxbfcode{\sphinxupquote{class\DUrole{w}{  }}}\sphinxcode{\sphinxupquote{dpav.}}\sphinxbfcode{\sphinxupquote{Window}}}{\emph{\DUrole{n}{arg1}\DUrole{p}{:}\DUrole{w}{  }\DUrole{n}{Optional\DUrole{p}{{[}}{\hyperref[\detokenize{dpav:dpav.vbuffer.VBuffer}]{\sphinxcrossref{dpav.vbuffer.VBuffer}}}\DUrole{p}{{]}}}\DUrole{w}{  }\DUrole{o}{=}\DUrole{w}{  }\DUrole{default_value}{None}}, \emph{\DUrole{n}{scale}\DUrole{p}{:}\DUrole{w}{  }\DUrole{n}{float}\DUrole{w}{  }\DUrole{o}{=}\DUrole{w}{  }\DUrole{default_value}{1.0}}}{}
\pysigstopsignatures
\sphinxAtStartPar
Bases: \sphinxcode{\sphinxupquote{object}}

\sphinxAtStartPar
Handles Window capabilites of Python Direct Platform
Functions:
\begin{quote}
\begin{description}
\item[{Constructor:}] \leavevmode
\sphinxAtStartPar
\_\_init\_\_()

\item[{Setters:}] \leavevmode
\sphinxAtStartPar
set\_scale(int/float)
set\_vbuffer(VBuffer/np.ndarray,optional:int)

\item[{Getters:}] \leavevmode
\sphinxAtStartPar
get\_mouse\_pos()

\item[{Misc Methods:}] \leavevmode
\sphinxAtStartPar
open()
is\_open()
close()
update()

\item[{Private Methods:}] \leavevmode
\sphinxAtStartPar
\_update\_events(pygame.event)
\_build\_events\_dict()
\_write\_to\_screen()

\end{description}
\end{quote}
\index{Public (dpav.Window attribute)@\spxentry{Public}\spxextra{dpav.Window attribute}}

\begin{fulllineitems}
\phantomsection\label{\detokenize{dpav:dpav.Window.Public}}
\pysigstartsignatures
\pysigline{\sphinxbfcode{\sphinxupquote{Public}}}
\pysigstopsignatures
\sphinxAtStartPar
vbuffer:     active VBuffer object
scale:       number that scales up/down the size of the screen
\begin{quote}

\sphinxAtStartPar
(1.0 is unscaled)
\end{quote}
\begin{description}
\item[{events:      dictionary of string:bool event pairs,}] \leavevmode\begin{description}
\item[{example:}] \leavevmode
\sphinxAtStartPar
“l\_shift”: True  \textendash{} left shift is pressed down
“l\_shift”: False \textendash{} left shift is not pressed

\end{description}

\item[{eventq:      list of active events that occured since last update cycle}] \leavevmode\begin{description}
\item[{example:}] \leavevmode
\sphinxAtStartPar
{[}‘l\_shift’, ‘mouse’, ‘a’, ‘q’{]}

\end{description}

\end{description}

\sphinxAtStartPar
debug\_flag:  boolean flag if window object should output debug info to log
open\_flag:   boolean flag for if the window is active

\end{fulllineitems}

\index{Private (dpav.Window attribute)@\spxentry{Private}\spxextra{dpav.Window attribute}}

\begin{fulllineitems}
\phantomsection\label{\detokenize{dpav:dpav.Window.Private}}
\pysigstartsignatures
\pysigline{\sphinxbfcode{\sphinxupquote{Private}}}
\pysigstopsignatures\begin{description}
\item[{\_keydict:    int:string PyGame event mapping. PyGame events identifiers are}] \leavevmode
\sphinxAtStartPar
stored as ints. This attribute is used by the public events
variable to map from PyGame’s integer:boolean pairs to
our string:boolean pairs

\item[{\_surfaces:   Two PyGame Surfaces for swapping to reflect vbuffer changes and}] \leavevmode
\sphinxAtStartPar
enable in\sphinxhyphen{}place nparray modification

\end{description}

\sphinxAtStartPar
\_screen:     PyGame.display object, used for viewing vbuffer attribute

\end{fulllineitems}

\index{close() (dpav.Window method)@\spxentry{close()}\spxextra{dpav.Window method}}

\begin{fulllineitems}
\phantomsection\label{\detokenize{dpav:dpav.Window.close}}
\pysigstartsignatures
\pysiglinewithargsret{\sphinxbfcode{\sphinxupquote{close}}}{}{{ $\rightarrow$ None}}
\pysigstopsignatures
\sphinxAtStartPar
Closes the active instance of a pygame window
\begin{quote}\begin{description}
\item[{Raises}] \leavevmode
\sphinxAtStartPar
\sphinxstyleliteralstrong{\sphinxupquote{RuntimeError}} \textendash{} no active pygame window instances exists

\end{description}\end{quote}

\end{fulllineitems}

\index{get\_mouse\_pos() (dpav.Window method)@\spxentry{get\_mouse\_pos()}\spxextra{dpav.Window method}}

\begin{fulllineitems}
\phantomsection\label{\detokenize{dpav:dpav.Window.get_mouse_pos}}
\pysigstartsignatures
\pysiglinewithargsret{\sphinxbfcode{\sphinxupquote{get\_mouse\_pos}}}{\emph{\DUrole{n}{) \sphinxhyphen{}\textgreater{} (\textless{}class \textquotesingle{}int\textquotesingle{}\textgreater{}}}, \emph{\DUrole{n}{\textless{}class \textquotesingle{}int\textquotesingle{}\textgreater{}}}}{}
\pysigstopsignatures
\sphinxAtStartPar
Returns the current mouse location with respect to the pygame window instance
\begin{quote}\begin{description}
\item[{Raises}] \leavevmode
\sphinxAtStartPar
\sphinxstyleliteralstrong{\sphinxupquote{Runtime Error}} \textendash{} no active pygame window instances exists

\end{description}\end{quote}

\end{fulllineitems}

\index{is\_open() (dpav.Window method)@\spxentry{is\_open()}\spxextra{dpav.Window method}}

\begin{fulllineitems}
\phantomsection\label{\detokenize{dpav:dpav.Window.is_open}}
\pysigstartsignatures
\pysiglinewithargsret{\sphinxbfcode{\sphinxupquote{is\_open}}}{}{{ $\rightarrow$ bool}}
\pysigstopsignatures
\sphinxAtStartPar
Updates events on every call, used to abstract out PyGame
display calls and event loop
\subsubsection*{Example}
\begin{description}
\item[{if window.is\_open():}] \leavevmode
\sphinxAtStartPar
\# your code here

\end{description}
\begin{quote}\begin{description}
\item[{Returns}] \leavevmode
\sphinxAtStartPar
boolean denoting if the window is currently open

\end{description}\end{quote}

\end{fulllineitems}

\index{open() (dpav.Window method)@\spxentry{open()}\spxextra{dpav.Window method}}

\begin{fulllineitems}
\phantomsection\label{\detokenize{dpav:dpav.Window.open}}
\pysigstartsignatures
\pysiglinewithargsret{\sphinxbfcode{\sphinxupquote{open}}}{}{{ $\rightarrow$ None}}
\pysigstopsignatures
\sphinxAtStartPar
Creates and runs pygame window in a new thread

\end{fulllineitems}

\index{set\_scale() (dpav.Window method)@\spxentry{set\_scale()}\spxextra{dpav.Window method}}

\begin{fulllineitems}
\phantomsection\label{\detokenize{dpav:dpav.Window.set_scale}}
\pysigstartsignatures
\pysiglinewithargsret{\sphinxbfcode{\sphinxupquote{set\_scale}}}{\emph{\DUrole{n}{scale}\DUrole{p}{:}\DUrole{w}{  }\DUrole{n}{float}}}{{ $\rightarrow$ None}}
\pysigstopsignatures
\sphinxAtStartPar
Sets the window scale

\end{fulllineitems}

\index{set\_vbuffer() (dpav.Window method)@\spxentry{set\_vbuffer()}\spxextra{dpav.Window method}}

\begin{fulllineitems}
\phantomsection\label{\detokenize{dpav:dpav.Window.set_vbuffer}}
\pysigstartsignatures
\pysiglinewithargsret{\sphinxbfcode{\sphinxupquote{set\_vbuffer}}}{\emph{\DUrole{n}{arg1}\DUrole{p}{:}\DUrole{w}{  }\DUrole{n}{{\hyperref[\detokenize{dpav:dpav.vbuffer.VBuffer}]{\sphinxcrossref{dpav.vbuffer.VBuffer}}}}}}{{ $\rightarrow$ None}}
\pysigstopsignatures
\sphinxAtStartPar
Sets the vbuffer/nparray object to display on screen
\begin{quote}\begin{description}
\item[{Parameters}] \leavevmode
\sphinxAtStartPar
\sphinxstyleliteralstrong{\sphinxupquote{arg1}} \textendash{} VBuffer/np.ndarray

\item[{Raises}] \leavevmode\begin{itemize}
\item {} 
\sphinxAtStartPar
\sphinxstyleliteralstrong{\sphinxupquote{TypeError}} \textendash{} arg1 VBuffer/np.ndarray type check

\item {} 
\sphinxAtStartPar
\sphinxstyleliteralstrong{\sphinxupquote{TypeError}} \textendash{} scale int/float type check

\end{itemize}

\end{description}\end{quote}

\end{fulllineitems}

\index{update() (dpav.Window method)@\spxentry{update()}\spxextra{dpav.Window method}}

\begin{fulllineitems}
\phantomsection\label{\detokenize{dpav:dpav.Window.update}}
\pysigstartsignatures
\pysiglinewithargsret{\sphinxbfcode{\sphinxupquote{update}}}{}{{ $\rightarrow$ None}}
\pysigstopsignatures
\sphinxAtStartPar
Pygame event abstraction, called at end of pygame loop.
Optional function if is\_open() is used
\begin{quote}\begin{description}
\item[{Raises}] \leavevmode
\sphinxAtStartPar
\sphinxstyleliteralstrong{\sphinxupquote{Runtime Error}} \textendash{} No active pygame window

\end{description}\end{quote}

\end{fulllineitems}


\end{fulllineitems}



\chapter{Indices and tables}
\label{\detokenize{index:indices-and-tables}}\begin{itemize}
\item {} 
\sphinxAtStartPar
\DUrole{xref,std,std-ref}{genindex}

\item {} 
\sphinxAtStartPar
\DUrole{xref,std,std-ref}{modindex}

\item {} 
\sphinxAtStartPar
\DUrole{xref,std,std-ref}{search}

\end{itemize}


\renewcommand{\indexname}{Python Module Index}
\begin{sphinxtheindex}
\let\bigletter\sphinxstyleindexlettergroup
\bigletter{d}
\item\relax\sphinxstyleindexentry{dpav}\sphinxstyleindexpageref{dpav:\detokenize{module-dpav}}
\item\relax\sphinxstyleindexentry{dpav.audio}\sphinxstyleindexpageref{dpav:\detokenize{module-dpav.audio}}
\item\relax\sphinxstyleindexentry{dpav.utility}\sphinxstyleindexpageref{dpav:\detokenize{module-dpav.utility}}
\item\relax\sphinxstyleindexentry{dpav.vbuffer}\sphinxstyleindexpageref{dpav:\detokenize{module-dpav.vbuffer}}
\item\relax\sphinxstyleindexentry{dpav.window}\sphinxstyleindexpageref{dpav:\detokenize{module-dpav.window}}
\end{sphinxtheindex}

\renewcommand{\indexname}{Index}
\printindex
\end{document}