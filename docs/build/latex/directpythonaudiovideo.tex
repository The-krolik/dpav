%% Generated by Sphinx.
\def\sphinxdocclass{report}
\documentclass[letterpaper,10pt,english,openany,oneside]{sphinxmanual}
\ifdefined\pdfpxdimen
   \let\sphinxpxdimen\pdfpxdimen\else\newdimen\sphinxpxdimen
\fi \sphinxpxdimen=.75bp\relax
\ifdefined\pdfimageresolution
    \pdfimageresolution= \numexpr \dimexpr1in\relax/\sphinxpxdimen\relax
\fi
%% let collapsible pdf bookmarks panel have high depth per default
\PassOptionsToPackage{bookmarksdepth=5}{hyperref}

\PassOptionsToPackage{warn}{textcomp}
\usepackage[utf8]{inputenc}
\ifdefined\DeclareUnicodeCharacter
% support both utf8 and utf8x syntaxes
  \ifdefined\DeclareUnicodeCharacterAsOptional
    \def\sphinxDUC#1{\DeclareUnicodeCharacter{"#1}}
  \else
    \let\sphinxDUC\DeclareUnicodeCharacter
  \fi
  \sphinxDUC{00A0}{\nobreakspace}
  \sphinxDUC{2500}{\sphinxunichar{2500}}
  \sphinxDUC{2502}{\sphinxunichar{2502}}
  \sphinxDUC{2514}{\sphinxunichar{2514}}
  \sphinxDUC{251C}{\sphinxunichar{251C}}
  \sphinxDUC{2572}{\textbackslash}
\fi
\usepackage{cmap}
\usepackage[T1]{fontenc}
\usepackage{amsmath,amssymb,amstext}
\usepackage{babel}



\usepackage{tgtermes}
\usepackage{tgheros}
\renewcommand{\ttdefault}{txtt}



\usepackage[Bjarne]{fncychap}
\usepackage{sphinx}

\fvset{fontsize=auto}
\usepackage{geometry}


% Include hyperref last.
\usepackage{hyperref}
% Fix anchor placement for figures with captions.
\usepackage{hypcap}% it must be loaded after hyperref.
% Set up styles of URL: it should be placed after hyperref.
\urlstyle{same}

\addto\captionsenglish{\renewcommand{\contentsname}{Contents:}}

\usepackage{sphinxmessages}
\setcounter{tocdepth}{1}



\title{Direct Python Audio/Video}
\date{May 01, 2022}
\release{0.0.1}
\author{Vibrant Labs}
\newcommand{\sphinxlogo}{\vbox{}}
\renewcommand{\releasename}{Release}
\makeindex
\begin{document}

\pagestyle{empty}
\sphinxmaketitle
\pagestyle{plain}
\sphinxtableofcontents
\pagestyle{normal}
\phantomsection\label{\detokenize{index::doc}}



\chapter{What is Direct Python Audio/Video?}
\label{\detokenize{fundamentals:what-is-direct-python-audio-video}}\label{\detokenize{fundamentals::doc}}
\sphinxAtStartPar
Direct Python Audio/Video is a library wrapping certain functionalities of Pygame that aims to give users a very simple, no\sphinxhyphen{}nonsense, direct feeling experience with basic audio and video manipulation. This library features the ability to craft basic waveforms and play them, as well as manipulate pixels in an image using 24\sphinxhyphen{}bit hex color codes, using no more than a few calls from our library. We abstract away technical aspects of interfacing with audio and video devices such as the need to maintain an event loop, in favor of straightforward calls that feel intuitive and beginner friendly.


\chapter{Installation}
\label{\detokenize{fundamentals:installation}}
\sphinxAtStartPar
This library may be installed by:

\sphinxAtStartPar
Cloning the repository:

\begin{sphinxVerbatim}[commandchars=\\\{\}]
\PYG{g+gp}{\PYGZgt{}\PYGZgt{}\PYGZgt{} }\PYG{n}{git} \PYG{n}{clone} \PYG{n}{https}\PYG{p}{:}\PYG{o}{/}\PYG{o}{/}\PYG{n}{github}\PYG{o}{.}\PYG{n}{com}\PYG{o}{/}\PYG{n}{The}\PYG{o}{\PYGZhy{}}\PYG{n}{krolik}\PYG{o}{/}\PYG{n}{dpav}
\end{sphinxVerbatim}

\sphinxAtStartPar
Then navigating to the cloned dpav folder and running:

\begin{sphinxVerbatim}[commandchars=\\\{\}]
\PYG{g+gp}{\PYGZgt{}\PYGZgt{}\PYGZgt{} }\PYG{n}{pip} \PYG{n}{install} \PYG{n}{dpav}
\end{sphinxVerbatim}


\chapter{Audio Class}
\label{\detokenize{fundamentals:audio-class}}
\sphinxAtStartPar
The Audio class is intended to provide basic sound capabilities focused around playing a constant tone for a desired duration in seconds. It supports playing one sound at a time with a waveform: sin, square, noise, saw, or triangle.
\begin{description}
\item[{To get started, there are three basic steps to play a tone:}] \leavevmode\begin{enumerate}
\sphinxsetlistlabels{\arabic}{enumi}{enumii}{}{.}%
\item {} 
\sphinxAtStartPar
Create an Audio class object

\item {} 
\sphinxAtStartPar
Call the play\_sound method with a frequency and duration (in seconds)

\item {} 
\sphinxAtStartPar
Use the wait\_for\_sound\_end. This maintains the process

\end{enumerate}

\end{description}
\sphinxSetupCaptionForVerbatim{Playing a sound}
\def\sphinxLiteralBlockLabel{\label{\detokenize{fundamentals:id2}}}
\begin{sphinxVerbatim}[commandchars=\\\{\}]
 \PYG{n}{mySound} \PYG{o}{=} \PYG{n}{dpp}\PYG{o}{.}\PYG{n}{Audio}\PYG{p}{(}\PYG{p}{)}
 \PYG{n}{frequency} \PYG{o}{=} \PYG{l+m+mi}{261}
 \PYG{n}{duration} \PYG{o}{=} \PYG{l+m+mi}{1}

 \PYG{n}{mySound}\PYG{o}{.}\PYG{n}{play\PYGZus{}sound}\PYG{p}{(}\PYG{n}{frequency}\PYG{p}{,} \PYG{n}{duration}\PYG{p}{)}
 \PYG{n}{mySound}\PYG{o}{.}\PYG{n}{wait\PYGZus{}for\PYGZus{}sound\PYGZus{}end}\PYG{p}{(}\PYG{p}{)}
\end{sphinxVerbatim}

\sphinxAtStartPar
If using audio alongside the Window class or within a while loop, the wait\_for\_sound\_end method is unnecessary.
\sphinxSetupCaptionForVerbatim{Using the play\_sound inside a while loop}
\def\sphinxLiteralBlockLabel{\label{\detokenize{fundamentals:id3}}}
\begin{sphinxVerbatim}[commandchars=\\\{\}]
 \PYG{n}{mySound} \PYG{o}{=} \PYG{n}{dpp}\PYG{o}{.}\PYG{n}{Audio}\PYG{p}{(}\PYG{p}{)}
 \PYG{n}{frequency} \PYG{o}{=} \PYG{l+m+mi}{261}
 \PYG{n}{duration} \PYG{o}{=} \PYG{l+m+mi}{1}

 \PYG{k}{while} \PYG{n}{window}\PYG{o}{.}\PYG{n}{is\PYGZus{}open}\PYG{p}{(}\PYG{p}{)}\PYG{p}{:}
     \PYG{n}{mySound}\PYG{o}{.}\PYG{n}{play\PYGZus{}sound}\PYG{p}{(}\PYG{n}{frequency}\PYG{p}{,} \PYG{n}{duration}\PYG{p}{)}
\end{sphinxVerbatim}

\sphinxAtStartPar
The utility function get\_note\_from\_string takes a music note, such as “C”, as a string and returns the frequency
\sphinxSetupCaptionForVerbatim{Using the utility function: get\_note\_from\_string}
\def\sphinxLiteralBlockLabel{\label{\detokenize{fundamentals:id4}}}
\begin{sphinxVerbatim}[commandchars=\\\{\}]
 \PYG{n}{mySound} \PYG{o}{=} \PYG{n}{dpp}\PYG{o}{.}\PYG{n}{Audio}\PYG{p}{(}\PYG{p}{)}
 \PYG{n}{frequency} \PYG{o}{=} \PYG{n}{dpp}\PYG{o}{.}\PYG{n}{get\PYGZus{}note\PYGZus{}from\PYGZus{}string}\PYG{p}{(}\PYG{l+s+s2}{\PYGZdq{}}\PYG{l+s+s2}{C}\PYG{l+s+s2}{\PYGZdq{}}\PYG{p}{,} \PYG{l+m+mi}{0}\PYG{p}{)}
 \PYG{n}{duration} \PYG{o}{=} \PYG{l+m+mi}{1}

 \PYG{n}{mySound}\PYG{o}{.}\PYG{n}{play\PYGZus{}sound}\PYG{p}{(}\PYG{n}{frequency}\PYG{p}{,} \PYG{n}{duration}\PYG{p}{)}
 \PYG{n}{mySound}\PYG{o}{.}\PYG{n}{wait\PYGZus{}for\PYGZus{}sound\PYGZus{}end}\PYG{p}{(}\PYG{p}{)}
\end{sphinxVerbatim}


\chapter{VBuffer Class}
\label{\detokenize{fundamentals:vbuffer-class}}
\sphinxAtStartPar
The VBuffer class operates as a 2\sphinxhyphen{}dimensional array of hex color values. This is the main data structure used for visualization within the Window class.


\section{Initialization}
\label{\detokenize{fundamentals:initialization}}\sphinxSetupCaptionForVerbatim{VBuffer initialization with dimensions 1920x1080}
\def\sphinxLiteralBlockLabel{\label{\detokenize{fundamentals:id5}}}
\begin{sphinxVerbatim}[commandchars=\\\{\}]
 \PYG{n}{vbuffer} \PYG{o}{=} \PYG{n}{dpp}\PYG{o}{.}\PYG{n}{VBuffer}\PYG{p}{(}\PYG{p}{(}\PYG{l+m+mi}{1920}\PYG{p}{,}\PYG{l+m+mi}{1080}\PYG{p}{)}\PYG{p}{)}
\end{sphinxVerbatim}
\sphinxSetupCaptionForVerbatim{VBuffer initialization with numpy array}
\def\sphinxLiteralBlockLabel{\label{\detokenize{fundamentals:id6}}}
\begin{sphinxVerbatim}[commandchars=\\\{\}]
 \PYG{n}{arr} \PYG{o}{=} \PYG{n}{np}\PYG{o}{.}\PYG{n}{zeros}\PYG{p}{(}\PYG{p}{(}\PYG{l+m+mi}{1920}\PYG{p}{,}\PYG{l+m+mi}{1080}\PYG{p}{)}\PYG{p}{)}
 \PYG{n}{vbuffer} \PYG{o}{=} \PYG{n}{dpp}\PYG{o}{.}\PYG{n}{VBuffer}\PYG{p}{(}\PYG{n}{arr}\PYG{p}{)}
\end{sphinxVerbatim}
\sphinxSetupCaptionForVerbatim{VBuffer default initializaion provides dimensions 800x600}
\def\sphinxLiteralBlockLabel{\label{\detokenize{fundamentals:id7}}}
\begin{sphinxVerbatim}[commandchars=\\\{\}]
 \PYG{n}{vbuffer} \PYG{o}{=} \PYG{n}{dpp}\PYG{o}{.}\PYG{n}{VBuffer}\PYG{p}{(}\PYG{p}{)}
\end{sphinxVerbatim}


\section{Modification}
\label{\detokenize{fundamentals:modification}}\sphinxSetupCaptionForVerbatim{Changing color of pixel to red at location: x=30, y=50}
\def\sphinxLiteralBlockLabel{\label{\detokenize{fundamentals:id8}}}
\begin{sphinxVerbatim}[commandchars=\\\{\}]
\PYG{n}{red} \PYG{o}{=} \PYG{l+m+mh}{0xFF0000}
\PYG{n}{vbuffer}\PYG{p}{[}\PYG{l+m+mi}{30}\PYG{p}{,}\PYG{l+m+mi}{50}\PYG{p}{]} \PYG{o}{=} \PYG{n}{red}
\end{sphinxVerbatim}
\sphinxSetupCaptionForVerbatim{Changing row 30 to red}
\def\sphinxLiteralBlockLabel{\label{\detokenize{fundamentals:id9}}}
\begin{sphinxVerbatim}[commandchars=\\\{\}]
\PYG{n}{red} \PYG{o}{=} \PYG{l+m+mh}{0xFF0000}
\PYG{n}{vbuffer}\PYG{p}{[}\PYG{l+m+mi}{30}\PYG{p}{,}\PYG{p}{:}\PYG{p}{]} \PYG{o}{=} \PYG{n}{red}
\end{sphinxVerbatim}
\sphinxSetupCaptionForVerbatim{Fill vbuffer object with color red}
\def\sphinxLiteralBlockLabel{\label{\detokenize{fundamentals:id10}}}
\begin{sphinxVerbatim}[commandchars=\\\{\}]
\PYG{n}{red} \PYG{o}{=} \PYG{l+m+mh}{0xFF0000}
\PYG{n}{vbuffer}\PYG{o}{.}\PYG{n}{fill}\PYG{p}{(}\PYG{n}{red}\PYG{p}{)}
\end{sphinxVerbatim}
\sphinxSetupCaptionForVerbatim{Clear vbuffer object with color red}
\def\sphinxLiteralBlockLabel{\label{\detokenize{fundamentals:id11}}}
\begin{sphinxVerbatim}[commandchars=\\\{\}]
\PYG{n}{vbuffer}\PYG{o}{.}\PYG{n}{clear}\PYG{p}{(}\PYG{p}{)}
\end{sphinxVerbatim}


\chapter{Window Class}
\label{\detokenize{fundamentals:window-class}}
\sphinxAtStartPar
The Window class is an abstraction of the PyGame library’s display and event handling. It is closely tied to the VBuffer class, using VBuffer objects as the primary data structure to hold the current image to display. An understanding of the VBuffer class may not be required for simple projects, such as those with static displays, but is recommended nonetheless, especially for more complicated use cases. Currently, only one window may be active at a time.


\section{Initialization}
\label{\detokenize{fundamentals:id1}}
\sphinxAtStartPar
Only one instance of the window class is needed throughout the lifetime of the program. Initialization of the object may be done in one of three ways, based upon the argument passed, or lack thereof. Passing a VBuffer object is the preferred method of initialization, however a 2\sphinxhyphen{}dimensional numpy array is also accepted, which will create the VBuffer for you. If neither are provided, the Window will create a default VBuffer with dimensions: (800,600).
\sphinxSetupCaptionForVerbatim{VBuffer initialization}
\def\sphinxLiteralBlockLabel{\label{\detokenize{fundamentals:id12}}}
\begin{sphinxVerbatim}[commandchars=\\\{\}]
 \PYG{n}{vbuffer} \PYG{o}{=} \PYG{n}{dpp}\PYG{o}{.}\PYG{n}{VBuffer}\PYG{p}{(}\PYG{p}{(}\PYG{l+m+mi}{1920}\PYG{p}{,}\PYG{l+m+mi}{1080}\PYG{p}{)}\PYG{p}{)}
 \PYG{n}{window} \PYG{o}{=} \PYG{n}{dpp}\PYG{o}{.}\PYG{n}{Window}\PYG{p}{(}\PYG{n}{vbuffer}\PYG{p}{)}
\end{sphinxVerbatim}
\sphinxSetupCaptionForVerbatim{Numpy array initialization}
\def\sphinxLiteralBlockLabel{\label{\detokenize{fundamentals:id13}}}
\begin{sphinxVerbatim}[commandchars=\\\{\}]
 \PYG{n}{vbuffer} \PYG{o}{=} \PYG{n}{numpy}\PYG{o}{.}\PYG{n}{zeros}\PYG{p}{(}\PYG{p}{(}\PYG{l+m+mi}{1920}\PYG{p}{,}\PYG{l+m+mi}{1080}\PYG{p}{)}\PYG{p}{)}
 \PYG{n}{window} \PYG{o}{=} \PYG{n}{dpp}\PYG{o}{.}\PYG{n}{Window}\PYG{p}{(}\PYG{n}{arr}\PYG{p}{)}
\end{sphinxVerbatim}
\sphinxSetupCaptionForVerbatim{Default initialization}
\def\sphinxLiteralBlockLabel{\label{\detokenize{fundamentals:id14}}}
\begin{sphinxVerbatim}[commandchars=\\\{\}]
 \PYG{n}{window} \PYG{o}{=} \PYG{n}{dpp}\PYG{o}{.}\PYG{n}{Window}\PYG{p}{(}\PYG{p}{)}
\end{sphinxVerbatim}


\section{Opening the Window}
\label{\detokenize{fundamentals:opening-the-window}}\begin{enumerate}
\sphinxsetlistlabels{\arabic}{enumi}{enumii}{}{.}%
\item {} 
\sphinxAtStartPar
Call open function

\item {} 
\sphinxAtStartPar
Construct while loop with is\_open function

\end{enumerate}

\begin{sphinxVerbatim}[commandchars=\\\{\}]
\PYG{n}{window}\PYG{o}{.}\PYG{n}{open}\PYG{p}{(}\PYG{p}{)}
\PYG{k}{while} \PYG{n}{window}\PYG{o}{.}\PYG{n}{is\PYGZus{}open}\PYG{p}{(}\PYG{p}{)}\PYG{p}{:}
\PYG{c+c1}{\PYGZsh{}\PYGZsh{}\PYGZsh{} your code here}
\end{sphinxVerbatim}

\sphinxAtStartPar
The open function creates and opens the display. The is\_open call maintains and updates the status of all events, as well as the display, on every call. The loop structure is required, as the display will become inactive otherwise.


\section{Scaling}
\label{\detokenize{fundamentals:scaling}}
\sphinxAtStartPar
The window may be scaled up or down in one of three ways:
\begin{enumerate}
\sphinxsetlistlabels{\arabic}{enumi}{enumii}{}{.}%
\item {} 
\sphinxAtStartPar
Provide a scale value to Window on initialization

\item {} 
\sphinxAtStartPar
Call the set\_scale function with the scale value

\item {} 
\sphinxAtStartPar
Directly modify the scale member

\end{enumerate}

\sphinxAtStartPar
The default scale value is 1.0. Reducing this value will reduce the size of the display, increasing it will increase the size of the display.

\sphinxAtStartPar
This feature can be useful. Such as:
creating a virtual canvas of dimensions (50,50). Scaling this up by a factor of 13 will provide display dimensions of (650,650), making it much easier to visualize any changes made.


\subsection{Events}
\label{\detokenize{fundamentals:events}}
\sphinxAtStartPar
Capturing events are the way which users utilize registered mouse clicks and key presses. Users have two ways to interface with these events


\section{Eventq List}
\label{\detokenize{fundamentals:eventq-list}}
\sphinxAtStartPar
The eventsq list will be most often used, as this structure is best for expressions that only need to register once per key press / mouse click. This list is updated on every iteration of the window loop, removing old events and adding new ones that have been registered. These events may be used by simply checking if a specific event is in the list.

\sphinxAtStartPar
Example of what may be held in the eventq after one iteration:
\sphinxSetupCaptionForVerbatim{Held in the eventq after one iteration example:}
\def\sphinxLiteralBlockLabel{\label{\detokenize{fundamentals:id15}}}
\begin{sphinxVerbatim}[commandchars=\\\{\}]
 \PYG{p}{[}\PYG{l+s+s2}{\PYGZdq{}}\PYG{l+s+s2}{a}\PYG{l+s+s2}{\PYGZdq{}}\PYG{p}{,} \PYG{l+s+s2}{\PYGZdq{}}\PYG{l+s+s2}{l\PYGZus{}shift}\PYG{l+s+s2}{\PYGZdq{}}\PYG{p}{]}
\end{sphinxVerbatim}


\section{Events Dictionary}
\label{\detokenize{fundamentals:events-dictionary}}
\sphinxAtStartPar
The events dictionary holds String:Boolean key:value pairs. The key indicates the event to check for, and value is a Boolean indicating if a key or the mouse is currently pressed. It is ideal for continuous expression calls while a key/mouse is held down. It is not recommended to utilize this interface unless incorporated with custom handling when only one expression call is required for an event trigger.
\sphinxSetupCaptionForVerbatim{Constantly printing to standard out while left\sphinxhyphen{}shift is held down}
\def\sphinxLiteralBlockLabel{\label{\detokenize{fundamentals:id16}}}
\begin{sphinxVerbatim}[commandchars=\\\{\}]
 \PYG{k}{while} \PYG{n}{window}\PYG{o}{.}\PYG{n}{is\PYGZus{}open}\PYG{p}{(}\PYG{p}{)}\PYG{p}{:}
     \PYG{k}{if} \PYG{n}{window}\PYG{o}{.}\PYG{n}{events}\PYG{p}{[}\PYG{l+s+s2}{\PYGZdq{}}\PYG{l+s+s2}{l\PYGZus{}shift}\PYG{l+s+s2}{\PYGZdq{}}\PYG{p}{]}\PYG{p}{:}
         \PYG{n+nb}{print}\PYG{p}{(}\PYG{l+s+s2}{\PYGZdq{}}\PYG{l+s+s2}{Left Shift is pressed DOWN!}\PYG{l+s+s2}{\PYGZdq{}}\PYG{p}{)}
\end{sphinxVerbatim}


\section{Mouse Position}
\label{\detokenize{fundamentals:mouse-position}}
\sphinxAtStartPar
Obtaining the current position of the mouse is done by calling the get\_mouse\_pos function. This will return a tuple of coordinates: (x , y). These coordinates are with respect to both the window, and the underlying VBuffer data structure.
\sphinxSetupCaptionForVerbatim{Setting pixel at mouse location to red}
\def\sphinxLiteralBlockLabel{\label{\detokenize{fundamentals:id17}}}
\begin{sphinxVerbatim}[commandchars=\\\{\}]
 \PYG{k}{if} \PYG{l+s+s2}{\PYGZdq{}}\PYG{l+s+s2}{mouse}\PYG{l+s+s2}{\PYGZdq{}} \PYG{o+ow}{in} \PYG{n}{window}\PYG{o}{.}\PYG{n}{eventq}\PYG{p}{:}
     \PYG{n}{red} \PYG{o}{=} \PYG{l+m+mh}{0xFF0000}
     \PYG{n}{pos} \PYG{o}{=} \PYG{n}{window}\PYG{o}{.}\PYG{n}{get\PYGZus{}mouse\PYGZus{}pos}\PYG{p}{(}\PYG{p}{)} \PYG{c+c1}{\PYGZsh{}get mouse position}
     \PYG{n}{window}\PYG{o}{.}\PYG{n}{vbuffer}\PYG{p}{[}\PYG{n}{pos}\PYG{p}{[}\PYG{l+m+mi}{0}\PYG{p}{]}\PYG{p}{,} \PYG{n}{pos}\PYG{p}{[}\PYG{l+m+mi}{1}\PYG{p}{]}\PYG{p}{]} \PYG{o}{=} \PYG{n}{red} \PYG{c+c1}{\PYGZsh{} set pixel at mouse (x,y) to red}
     \PYG{n+nb}{print}\PYG{p}{(}\PYG{l+s+sa}{f}\PYG{l+s+s2}{\PYGZdq{}}\PYG{l+s+s2}{Color at }\PYG{l+s+si}{\PYGZob{}}\PYG{n}{pos}\PYG{l+s+si}{\PYGZcb{}}\PYG{l+s+s2}{ changed to Red}\PYG{l+s+s2}{\PYGZdq{}}\PYG{p}{)}
\end{sphinxVerbatim}


\chapter{Source Documentation}
\label{\detokenize{dpav:source-documentation}}\label{\detokenize{dpav::doc}}

\section{Audio}
\label{\detokenize{dpav:audio}}

\begin{fulllineitems}
\pysigline{\sphinxbfcode{\sphinxupquote{class }}\sphinxcode{\sphinxupquote{dpav.audio.}}\sphinxbfcode{\sphinxupquote{Audio}}}
\sphinxAtStartPar
Bases: \sphinxcode{\sphinxupquote{object}}

\sphinxAtStartPar
Handles Audio capabilities of Python Direct Platform.
\begin{description}
\item[{Functions:}] \leavevmode\begin{description}
\item[{Constructor:}] \leavevmode
\sphinxAtStartPar
\_\_init\_\_()

\item[{Functions:}] \leavevmode\begin{description}
\item[{play\_sound(Hz, length)}] \leavevmode\begin{description}
\item[{If audio buffer is set:}] \leavevmode
\sphinxAtStartPar
play\_sound()

\end{description}

\end{description}

\sphinxAtStartPar
play\_sample(string\_name\_of\_wav\_file)

\item[{Setters:}] \leavevmode
\sphinxAtStartPar
set\_audio\_buffer(numpyarray)
set\_audio\_device(int)
set\_waveform(waveform)

\item[{Getters:}] \leavevmode
\sphinxAtStartPar
get\_bit\_number()\sphinxhyphen{}\textgreater{}int
get\_sample\_rate()\sphinxhyphen{}\textgreater{}int
get\_audio\_buffer()
get\_audio\_device()\sphinxhyphen{}\textgreater{}Returns int corresponding to audio device

\item[{Misc:}] \leavevmode
\sphinxAtStartPar
list\_audio\_devices()
wait\_for\_sound\_end()

\end{description}

\end{description}


\begin{fulllineitems}
\pysiglinewithargsret{\sphinxbfcode{\sphinxupquote{get\_audio\_buffer}}}{}{}
\sphinxAtStartPar
Returns the audio buffer of the Audio class
\begin{description}
\item[{Description:}] \leavevmode
\sphinxAtStartPar
This will return none if the audio buffer has not been set by the set\_audio\_buffer method.

\sphinxAtStartPar
audioobject.get\_audio\_buffer()

\end{description}
\begin{quote}\begin{description}
\item[{Parameters}] \leavevmode
\sphinxAtStartPar
\sphinxstyleliteralstrong{\sphinxupquote{None}} \textendash{} 

\item[{Returns}] \leavevmode
\sphinxAtStartPar
numpy array

\item[{Return type}] \leavevmode
\sphinxAtStartPar
self.\_audio\_buffer

\end{description}\end{quote}

\end{fulllineitems}



\begin{fulllineitems}
\pysiglinewithargsret{\sphinxbfcode{\sphinxupquote{get\_audio\_device}}}{}{{ $\rightarrow$ int}}
\sphinxAtStartPar
Gets the current audio device number of the Audio Class
\begin{description}
\item[{Description:}] \leavevmode
\sphinxAtStartPar
Assuming audioobject.set\_audio\_device(2) is called,
audioobject.get\_audio\_device() would return 2 {[}index of audio device in audioobject.list\_audio\_devices(){]}

\end{description}
\begin{quote}\begin{description}
\item[{Parameters}] \leavevmode
\sphinxAtStartPar
\sphinxstyleliteralstrong{\sphinxupquote{None}} \textendash{} 

\end{description}\end{quote}
\begin{description}
\item[{Returns}] \leavevmode
\sphinxAtStartPar
self.\_audio\_device: int value

\end{description}
\subsubsection*{Notes}

\sphinxAtStartPar
Returns the integer value of the device not the device name

\end{fulllineitems}



\begin{fulllineitems}
\pysiglinewithargsret{\sphinxbfcode{\sphinxupquote{get\_bit\_number}}}{}{{ $\rightarrow$ int}}
\sphinxAtStartPar
Gets the bit rate of the Audio class
\begin{description}
\item[{Description:}] \leavevmode
\sphinxAtStartPar
Bit rate currently locked to 16 bits

\end{description}
\begin{quote}\begin{description}
\item[{Parameters}] \leavevmode
\sphinxAtStartPar
\sphinxstyleliteralstrong{\sphinxupquote{None}} \textendash{} 

\item[{Returns}] \leavevmode
\sphinxAtStartPar
The bit rate of the Audio class \sphinxhyphen{} int value

\item[{Return type}] \leavevmode
\sphinxAtStartPar
self.\_bit\_number

\end{description}\end{quote}

\end{fulllineitems}



\begin{fulllineitems}
\pysiglinewithargsret{\sphinxbfcode{\sphinxupquote{get\_sample\_rate}}}{}{{ $\rightarrow$ int}}
\sphinxAtStartPar
Gets the sample rate of the Audio class.
\begin{description}
\item[{Description:}] \leavevmode
\sphinxAtStartPar
Sample rate is currently locked to 44100

\end{description}
\begin{quote}\begin{description}
\item[{Parameters}] \leavevmode
\sphinxAtStartPar
\sphinxstyleliteralstrong{\sphinxupquote{None}} \textendash{} 

\item[{Returns}] \leavevmode
\sphinxAtStartPar
The sample rate of the audioClass \sphinxhyphen{} int value

\item[{Return type}] \leavevmode
\sphinxAtStartPar
self.\_sample\_rate

\end{description}\end{quote}

\end{fulllineitems}



\begin{fulllineitems}
\pysiglinewithargsret{\sphinxbfcode{\sphinxupquote{list\_audio\_devices}}}{}{{ $\rightarrow$ None}}
\sphinxAtStartPar
Lists the output devices on your system and adds to list self.\_devices
\begin{description}
\item[{Description:}] \leavevmode
\sphinxAtStartPar
Run this function before using set\_audio\_device() to add devices to the list devices

\sphinxAtStartPar
audioobject.list\_audio\_devices()
0 Speakers (Realtek(R) Audio)
1 VGA248 (2\sphinxhyphen{}NVIDIA High Def Audio)
2 Speakers (HyperX Cloud II Wireless)

\end{description}
\begin{quote}\begin{description}
\item[{Parameters}] \leavevmode
\sphinxAtStartPar
\sphinxstyleliteralstrong{\sphinxupquote{None}} \textendash{} 

\item[{Returns}] \leavevmode
\sphinxAtStartPar
None

\end{description}\end{quote}

\end{fulllineitems}



\begin{fulllineitems}
\pysiglinewithargsret{\sphinxbfcode{\sphinxupquote{play\_sample}}}{\emph{\DUrole{n}{sample\_name}\DUrole{p}{:} \DUrole{n}{str}}}{{ $\rightarrow$ None}}
\sphinxAtStartPar
Plays sounds that are wav, ogg or mp3 files.
\begin{description}
\item[{Description:}] \leavevmode
\sphinxAtStartPar
audioobject.play\_sample(mypath.mp3) would play sounds from the file mypath.mp3

\end{description}
\begin{quote}\begin{description}
\item[{Parameters}] \leavevmode
\sphinxAtStartPar
\sphinxstyleliteralstrong{\sphinxupquote{sample\_name}} \textendash{} String path or name of sound

\item[{Returns}] \leavevmode
\sphinxAtStartPar
None

\end{description}\end{quote}

\end{fulllineitems}



\begin{fulllineitems}
\pysiglinewithargsret{\sphinxbfcode{\sphinxupquote{play\_sound}}}{\emph{\DUrole{n}{input\_frequency}\DUrole{o}{=}\DUrole{default_value}{0}}, \emph{\DUrole{n}{input\_duration}\DUrole{o}{=}\DUrole{default_value}{0}}}{{ $\rightarrow$ None}}
\sphinxAtStartPar
Primary sound playing method of the audio class.
\begin{description}
\item[{Description:}] \leavevmode
\sphinxAtStartPar
Play sounds directly from this function
Need to run set\_audio\_device() or will default to the default audio device
You can use set\_waveform to change the type.
play\_sound is somewhat overloaded to where if you have an audioBuffer set using set\_audio\_buffer, you can call play\_sound()
\begin{quote}

\sphinxAtStartPar
and it will play whatever that audio\_buffer is e.g. wav files
Example in examples/custombuffer.py
\end{quote}

\sphinxAtStartPar
play\_sound(440, 1) would play an A note for one second with the sin waveform set.

\end{description}
\begin{quote}\begin{description}
\item[{Parameters}] \leavevmode\begin{itemize}
\item {} 
\sphinxAtStartPar
\sphinxstyleliteralstrong{\sphinxupquote{input\_frequency}} \textendash{} int value \sphinxhyphen{} input frequency in Hz

\item {} 
\sphinxAtStartPar
\sphinxstyleliteralstrong{\sphinxupquote{input\_duration}} \textendash{} int value \sphinxhyphen{} duration in seconds

\end{itemize}

\item[{Raises}] \leavevmode
\sphinxAtStartPar
\sphinxstyleliteralstrong{\sphinxupquote{TypeError}} \textendash{} If input\_duration not a number, or \textless{} 0

\item[{Returns}] \leavevmode
\sphinxAtStartPar
None

\end{description}\end{quote}

\end{fulllineitems}



\begin{fulllineitems}
\pysiglinewithargsret{\sphinxbfcode{\sphinxupquote{set\_audio\_buffer}}}{\emph{\DUrole{n}{ab}}}{{ $\rightarrow$ None}}
\sphinxAtStartPar
Sets the audio buffer of the Audio Class.
\begin{description}
\item[{Description:}] \leavevmode
\sphinxAtStartPar
The audio buffer needs to have two rows so that way stereo works as intended.
You can set the audio buffer to wav file data by fetching numpy arrays using wav or scipy,
however only 16 bit waves are supported. This process can be seen in custom\_buffer.py w/ the
utility function sixteenWavtoRawData
\begin{description}
\item[{Examples:}] \leavevmode
\sphinxAtStartPar
\# 44100 = sample rate
\# 32767 is 2 \textasciicircum{} (our bit depth \sphinxhyphen{}1)\sphinxhyphen{}1 and is essentially the number of samples per time stamp
\# 260 and 290 are our tones in hz
\# Below generates a buffer 1 second long of sin wave data\sphinxhyphen{}identical to the method used in house
data = numpy.zeros((44100, 2), dtype=numpy.int16)
for s in range(44100):
\begin{quote}

\sphinxAtStartPar
t = float(s) / 44100
data{[}s{]}{[}0{]} = int(round(32767 * math.sin(2 * math.pi * 260 * t)))
data{[}s{]}{[}1{]} = int(round(32767 * math.sin(2 * math.pi * 290 * t)))
\end{quote}

\sphinxAtStartPar
audioobject.set\_audio\_buffer(data)

\end{description}

\end{description}
\begin{quote}\begin{description}
\item[{Parameters}] \leavevmode
\sphinxAtStartPar
\sphinxstyleliteralstrong{\sphinxupquote{ab}} \textendash{} numpy array of shape(samples, channels) e.g. ab{[}44100{]}{[}2{]}

\item[{Returns}] \leavevmode
\sphinxAtStartPar
None

\end{description}\end{quote}

\end{fulllineitems}



\begin{fulllineitems}
\pysiglinewithargsret{\sphinxbfcode{\sphinxupquote{set\_audio\_device}}}{\emph{\DUrole{n}{device}\DUrole{p}{:} \DUrole{n}{int}}}{{ $\rightarrow$ int}}
\sphinxAtStartPar
Sets the current audio device of the Audio class.
\begin{description}
\item[{Description:}] \leavevmode
\sphinxAtStartPar
This can only be set ONCE per instance. To change devices, del the current instance
set the new device, and continue
This needs to be run after list\_audio\_device() in order to see list of audio devices
If not run the device will default to the current device being used by the machine

\sphinxAtStartPar
audioobject.set\_audio\_device(2)
Based on example in list\_audio\_devices() this would change the device to Speakers (HyperX Cloud II Wireless)

\end{description}
\begin{quote}\begin{description}
\item[{Parameters}] \leavevmode
\sphinxAtStartPar
\sphinxstyleliteralstrong{\sphinxupquote{device}} \textendash{} int value \sphinxhyphen{} see all int values for each device by running list\_audio\_devices()

\item[{Returns}] \leavevmode
\sphinxAtStartPar
None

\end{description}\end{quote}

\end{fulllineitems}



\begin{fulllineitems}
\pysiglinewithargsret{\sphinxbfcode{\sphinxupquote{set\_waveform}}}{\emph{\DUrole{n}{wave}}}{{ $\rightarrow$ None}}
\sphinxAtStartPar
Sets the expression governing the wave form playing
\begin{description}
\item[{Description:}] \leavevmode
\sphinxAtStartPar
play\_audio uses this in buffer generation

\sphinxAtStartPar
audioobject.set\_waveform(object.wave\_table.sin)
This would change to the waveform sin contained in the wave\_table class
The wave functions need to take in a input frequency as well as a timestep parameter
to solve for a particular frequency at a given time step. See wave\_table for an example of this.

\end{description}
\begin{quote}\begin{description}
\item[{Parameters}] \leavevmode
\sphinxAtStartPar
\sphinxstyleliteralstrong{\sphinxupquote{Wave}} \textendash{} takes a mathematical expression function ‘pointer’ in the form of f(inputfreq, timestep)

\item[{Returns}] \leavevmode
\sphinxAtStartPar
None

\end{description}\end{quote}

\end{fulllineitems}



\begin{fulllineitems}
\pysiglinewithargsret{\sphinxbfcode{\sphinxupquote{wait\_for\_sound\_end}}}{}{}
\sphinxAtStartPar
Function call that is placed at the end of scripts without a pygame window instance so sounds play to their full duration without a
\begin{description}
\item[{Description:}] \leavevmode
\sphinxAtStartPar
Placed at the end of python files that do not have loops. Otherwise, sounds would be cut off prematurely.
\begin{description}
\item[{Example:}] \leavevmode
\sphinxAtStartPar
play\_sound(440, 10)
wait\_for\_sound\_end() \# This prevents the process from closing out before the sound ends.

\end{description}

\end{description}
\begin{quote}\begin{description}
\item[{Parameters}] \leavevmode
\sphinxAtStartPar
\sphinxstyleliteralstrong{\sphinxupquote{None}} \textendash{} 

\item[{Returns}] \leavevmode
\sphinxAtStartPar
None

\end{description}\end{quote}

\sphinxAtStartPar
Notes:

\end{fulllineitems}


\end{fulllineitems}



\begin{fulllineitems}
\pysigline{\sphinxbfcode{\sphinxupquote{class }}\sphinxcode{\sphinxupquote{dpav.audio.}}\sphinxbfcode{\sphinxupquote{wave\_table}}}
\sphinxAtStartPar
Bases: \sphinxcode{\sphinxupquote{object}}

\sphinxAtStartPar
This is a class holding waveforms for usage with the play\_sound method.
\begin{description}
\item[{There are 5 waveforms:}] \leavevmode
\sphinxAtStartPar
sin
saw
square
noise
triangle

\end{description}
\subsubsection*{Example}

\sphinxAtStartPar
waves = wave\_table()
sinefunc = waves.sin


\begin{fulllineitems}
\pysiglinewithargsret{\sphinxbfcode{\sphinxupquote{noise}}}{\emph{\DUrole{n}{input\_frequency}}, \emph{\DUrole{n}{t}}}{}
\sphinxAtStartPar
Random white noise
\begin{description}
\item[{Description:}] \leavevmode
\sphinxAtStartPar
Warning: VERY LOUD

\end{description}
\begin{quote}\begin{description}
\item[{Parameters}] \leavevmode\begin{itemize}
\item {} 
\sphinxAtStartPar
\sphinxstyleliteralstrong{\sphinxupquote{input\_frequency}} \textendash{} value in Hz at timestep t

\item {} 
\sphinxAtStartPar
\sphinxstyleliteralstrong{\sphinxupquote{t}} \textendash{} timestep

\end{itemize}

\item[{Returns}] \leavevmode
\sphinxAtStartPar


\item[{Return type}] \leavevmode
\sphinxAtStartPar
random.random() * input\_frequency * t

\end{description}\end{quote}

\end{fulllineitems}



\begin{fulllineitems}
\pysiglinewithargsret{\sphinxbfcode{\sphinxupquote{saw}}}{\emph{\DUrole{n}{input\_frequency}}, \emph{\DUrole{n}{t}}}{}
\sphinxAtStartPar
Saw wave
\begin{quote}\begin{description}
\item[{Parameters}] \leavevmode\begin{itemize}
\item {} 
\sphinxAtStartPar
\sphinxstyleliteralstrong{\sphinxupquote{input\_frequency}} \textendash{} value in Hz at timestep t

\item {} 
\sphinxAtStartPar
\sphinxstyleliteralstrong{\sphinxupquote{t}} \textendash{} timestep

\end{itemize}

\item[{Returns}] \leavevmode
\sphinxAtStartPar


\item[{Return type}] \leavevmode
\sphinxAtStartPar
t * input\_frequency \sphinxhyphen{} math.floor(t * input\_frequency)

\end{description}\end{quote}

\end{fulllineitems}



\begin{fulllineitems}
\pysiglinewithargsret{\sphinxbfcode{\sphinxupquote{sin}}}{\emph{\DUrole{n}{input\_frequency}}, \emph{\DUrole{n}{t}}}{}
\sphinxAtStartPar
Sin wave form, default for libary
\begin{quote}\begin{description}
\item[{Parameters}] \leavevmode\begin{itemize}
\item {} 
\sphinxAtStartPar
\sphinxstyleliteralstrong{\sphinxupquote{input\_frequency}} \textendash{} value in Hz at timestep t

\item {} 
\sphinxAtStartPar
\sphinxstyleliteralstrong{\sphinxupquote{t}} \textendash{} timestep

\end{itemize}

\item[{Returns}] \leavevmode
\sphinxAtStartPar


\item[{Return type}] \leavevmode
\sphinxAtStartPar
math.sin(2 * math.pi * input\_frequency * t)

\end{description}\end{quote}

\end{fulllineitems}



\begin{fulllineitems}
\pysiglinewithargsret{\sphinxbfcode{\sphinxupquote{square}}}{\emph{\DUrole{n}{input\_frequency}}, \emph{\DUrole{n}{t}}}{}
\sphinxAtStartPar
Square wave form
\begin{quote}\begin{description}
\item[{Parameters}] \leavevmode\begin{itemize}
\item {} 
\sphinxAtStartPar
\sphinxstyleliteralstrong{\sphinxupquote{input\_frequency}} \textendash{} value in Hz at timestep t

\item {} 
\sphinxAtStartPar
\sphinxstyleliteralstrong{\sphinxupquote{t}} \textendash{} timestep

\end{itemize}

\item[{Returns}] \leavevmode
\sphinxAtStartPar


\item[{Return type}] \leavevmode
\sphinxAtStartPar
round(math.sin(2 * math.pi * input\_frequency * t))

\end{description}\end{quote}

\end{fulllineitems}



\begin{fulllineitems}
\pysiglinewithargsret{\sphinxbfcode{\sphinxupquote{triangle}}}{\emph{\DUrole{n}{input\_frequency}}, \emph{\DUrole{n}{t}}}{}
\sphinxAtStartPar
Triangle wave, similar in sound to saw + sin together
\begin{quote}\begin{description}
\item[{Parameters}] \leavevmode\begin{itemize}
\item {} 
\sphinxAtStartPar
\sphinxstyleliteralstrong{\sphinxupquote{input\_frequency}} \textendash{} value in Hz at timestep t

\item {} 
\sphinxAtStartPar
\sphinxstyleliteralstrong{\sphinxupquote{t}} \textendash{} timestep

\end{itemize}

\item[{Returns}] \leavevmode
\sphinxAtStartPar


\item[{Return type}] \leavevmode
\sphinxAtStartPar
2 * abs((t * input\_frequency) / 1 \sphinxhyphen{} math.floor(((t * input\_frequency) / 1) + 0.5))

\end{description}\end{quote}

\end{fulllineitems}


\end{fulllineitems}



\section{VBuffer}
\label{\detokenize{dpav:vbuffer}}

\begin{fulllineitems}
\pysiglinewithargsret{\sphinxbfcode{\sphinxupquote{class }}\sphinxcode{\sphinxupquote{dpav.vbuffer.}}\sphinxbfcode{\sphinxupquote{VBuffer}}}{\emph{\DUrole{n}{arg1}\DUrole{p}{:} \DUrole{n}{tuple} \DUrole{o}{=} \DUrole{default_value}{(800, 600)}}}{}
\sphinxAtStartPar
Bases: \sphinxcode{\sphinxupquote{object}}

\sphinxAtStartPar
Visual buffer for the Python Direct Platform

\sphinxAtStartPar
Holds a 2D array of hex color values. Each element represents a pixel,
whose coordinates are its index. VBuffer can be loaded and displayed by
the window class.
\begin{quote}\begin{description}
\item[{Parameters}] \leavevmode
\sphinxAtStartPar
\sphinxstyleliteralstrong{\sphinxupquote{arg1}} (\sphinxstyleliteralemphasis{\sphinxupquote{\{}}\sphinxstyleliteralemphasis{\sphinxupquote{(}}\sphinxstyleliteralemphasis{\sphinxupquote{int}}\sphinxstyleliteralemphasis{\sphinxupquote{, }}\sphinxstyleliteralemphasis{\sphinxupquote{int}}\sphinxstyleliteralemphasis{\sphinxupquote{)}}\sphinxstyleliteralemphasis{\sphinxupquote{|}}\sphinxstyleliteralemphasis{\sphinxupquote{np.ndarray}}\sphinxstyleliteralemphasis{\sphinxupquote{(}}\sphinxstyleliteralemphasis{\sphinxupquote{int}}\sphinxstyleliteralemphasis{\sphinxupquote{, }}\sphinxstyleliteralemphasis{\sphinxupquote{int}}\sphinxstyleliteralemphasis{\sphinxupquote{)}}\sphinxstyleliteralemphasis{\sphinxupquote{\}}}) \textendash{} 
\sphinxAtStartPar
Either array dimensions or a 2\sphinxhyphen{}dimensional numpy array of integers

\sphinxAtStartPar
If dimensions, will create zeroed\sphinxhyphen{}out 2D array of the selected
dimensions. Defaults to 800x600.

\sphinxAtStartPar
If numpy array, will set buffer to the contents of that array.


\end{description}\end{quote}


\begin{fulllineitems}
\pysigline{\sphinxbfcode{\sphinxupquote{Constructor:}}}
\sphinxAtStartPar
\_\_init\_\_(self, arg1=(800, 600)) \sphinxhyphen{}\textgreater{} None

\end{fulllineitems}



\begin{fulllineitems}
\pysigline{\sphinxbfcode{\sphinxupquote{Overloads:}}}
\sphinxAtStartPar
\_\_getitem\_\_(self, idx) \sphinxhyphen{}\textgreater{} int
\_\_setitem\_\_(self, idx, val) \sphinxhyphen{}\textgreater{} None
\_\_len\_\_(self) \sphinxhyphen{}\textgreater{} int

\end{fulllineitems}



\begin{fulllineitems}
\pysigline{\sphinxbfcode{\sphinxupquote{properties:}}}~\begin{description}
\item[{getter:}] \leavevmode
\sphinxAtStartPar
dimensions(self) \sphinxhyphen{}\textgreater{} (int, int)

\item[{setter:}] \leavevmode
\sphinxAtStartPar
dimensions(self, val) \sphinxhyphen{}\textgreater{} None

\end{description}

\end{fulllineitems}



\begin{fulllineitems}
\pysigline{\sphinxbfcode{\sphinxupquote{Setter:}}}
\sphinxAtStartPar
write\_pixel(self, coords, val) \sphinxhyphen{}\textgreater{} None
set\_buffer(self, buf) \sphinxhyphen{}\textgreater{} None
clear(self) \sphinxhyphen{}\textgreater{} None
fill(self, color: int) \sphinxhyphen{}\textgreater{} None

\end{fulllineitems}



\begin{fulllineitems}
\pysigline{\sphinxbfcode{\sphinxupquote{Getters:}}}
\sphinxAtStartPar
get\_pixel(self, coords) \sphinxhyphen{}\textgreater{} int
get\_dimensions(self) \sphinxhyphen{}\textgreater{} (int, int)

\end{fulllineitems}



\begin{fulllineitems}
\pysigline{\sphinxbfcode{\sphinxupquote{File~I/O:}}}
\sphinxAtStartPar
save\_buffer\_to\_file(self, filename) \sphinxhyphen{}\textgreater{} None
load\_buffer\_from\_file(self, filename) \sphinxhyphen{}\textgreater{} None

\end{fulllineitems}



\begin{fulllineitems}
\pysigline{\sphinxbfcode{\sphinxupquote{Error~Checking:}}}
\sphinxAtStartPar
\_check\_numpy\_arr(self,arg1,arg\_name,method\_name) \sphinxhyphen{}\textgreater{} None
\_check\_coord\_type(self, coords, arg\_name, method\_name) \sphinxhyphen{}\textgreater{} None
\_check\_coord\_vals(self, x, y, method\_name) \sphinxhyphen{}\textgreater{} None

\end{fulllineitems}



\begin{fulllineitems}
\pysiglinewithargsret{\sphinxbfcode{\sphinxupquote{clear}}}{}{{ $\rightarrow$ None}}
\sphinxAtStartPar
Set every pixel in buffer to 0 (hex value for black).

\end{fulllineitems}



\begin{fulllineitems}
\pysigline{\sphinxbfcode{\sphinxupquote{property }}\sphinxbfcode{\sphinxupquote{dimensions}}\sphinxbfcode{\sphinxupquote{: tuple}}}
\sphinxAtStartPar
Return dimensions of buffer.

\end{fulllineitems}



\begin{fulllineitems}
\pysiglinewithargsret{\sphinxbfcode{\sphinxupquote{fill}}}{\emph{\DUrole{n}{color}\DUrole{p}{:} \DUrole{n}{int}}}{{ $\rightarrow$ None}}
\sphinxAtStartPar
Set every pixel in the buffer to a given color.
\begin{quote}\begin{description}
\item[{Parameters}] \leavevmode
\sphinxAtStartPar
\sphinxstyleliteralstrong{\sphinxupquote{color}} (\sphinxstyleliteralemphasis{\sphinxupquote{Hex color code}}) \textendash{} 

\end{description}\end{quote}

\end{fulllineitems}



\begin{fulllineitems}
\pysiglinewithargsret{\sphinxbfcode{\sphinxupquote{get\_dimensions}}}{}{{ $\rightarrow$ tuple}}
\sphinxAtStartPar
Return dimensions of visual buffer array.

\end{fulllineitems}



\begin{fulllineitems}
\pysiglinewithargsret{\sphinxbfcode{\sphinxupquote{get\_pixel}}}{\emph{\DUrole{n}{coords}\DUrole{p}{:} \DUrole{n}{tuple}}}{{ $\rightarrow$ int}}
\sphinxAtStartPar
Return color value of chosen pixel.
\begin{quote}\begin{description}
\item[{Parameters}] \leavevmode
\sphinxAtStartPar
\sphinxstyleliteralstrong{\sphinxupquote{coords}} (\sphinxstyleliteralemphasis{\sphinxupquote{2\sphinxhyphen{}tuple}}\sphinxstyleliteralemphasis{\sphinxupquote{ or }}\sphinxstyleliteralemphasis{\sphinxupquote{list containing first and second index of pixel}}) \textendash{} 

\end{description}\end{quote}

\end{fulllineitems}



\begin{fulllineitems}
\pysiglinewithargsret{\sphinxbfcode{\sphinxupquote{load\_buffer\_from\_file}}}{\emph{\DUrole{n}{filename}\DUrole{p}{:} \DUrole{n}{str}}}{{ $\rightarrow$ None}}
\sphinxAtStartPar
Load binary file storing buffer contents, and write it to buffer.
\begin{quote}\begin{description}
\item[{Parameters}] \leavevmode
\sphinxAtStartPar
\sphinxstyleliteralstrong{\sphinxupquote{filename}} (\sphinxstyleliteralemphasis{\sphinxupquote{Path to a binary file containing numpy array data}}) \textendash{} 

\end{description}\end{quote}

\end{fulllineitems}



\begin{fulllineitems}
\pysiglinewithargsret{\sphinxbfcode{\sphinxupquote{save\_buffer\_to\_file}}}{\emph{\DUrole{n}{filename}\DUrole{p}{:} \DUrole{n}{str}}}{{ $\rightarrow$ None}}
\sphinxAtStartPar
Save contents of buffer to a binary file.
\begin{quote}\begin{description}
\item[{Parameters}] \leavevmode
\sphinxAtStartPar
\sphinxstyleliteralstrong{\sphinxupquote{filename}} (\sphinxstyleliteralemphasis{\sphinxupquote{The path and name of the file to write to}}) \textendash{} 

\end{description}\end{quote}

\end{fulllineitems}



\begin{fulllineitems}
\pysiglinewithargsret{\sphinxbfcode{\sphinxupquote{set\_buffer}}}{\emph{\DUrole{n}{buf}\DUrole{p}{:} \DUrole{n}{numpy.ndarray}}}{{ $\rightarrow$ None}}
\sphinxAtStartPar
Set the visual buffer to equal a provided 2D array of pixels.
\begin{quote}\begin{description}
\item[{Parameters}] \leavevmode
\sphinxAtStartPar
\sphinxstyleliteralstrong{\sphinxupquote{buf}} (\sphinxstyleliteralemphasis{\sphinxupquote{A 2\sphinxhyphen{}dimensional numpy array of integer color values}}) \textendash{} 

\end{description}\end{quote}

\end{fulllineitems}



\begin{fulllineitems}
\pysiglinewithargsret{\sphinxbfcode{\sphinxupquote{write\_pixel}}}{\emph{\DUrole{n}{coords}\DUrole{p}{:} \DUrole{n}{tuple}}, \emph{\DUrole{n}{val}\DUrole{p}{:} \DUrole{n}{int}}}{{ $\rightarrow$ None}}
\sphinxAtStartPar
Sets pixel at specified coordinates to specified color.

\sphinxAtStartPar
Sets pixel at coordinates coords in buffer to hex value val
\begin{quote}\begin{description}
\item[{Parameters}] \leavevmode\begin{itemize}
\item {} 
\sphinxAtStartPar
\sphinxstyleliteralstrong{\sphinxupquote{coords}} (\sphinxstyleliteralemphasis{\sphinxupquote{Pixel coordinates}}\sphinxstyleliteralemphasis{\sphinxupquote{ (}}\sphinxstyleliteralemphasis{\sphinxupquote{an X and a Y}}\sphinxstyleliteralemphasis{\sphinxupquote{)}}) \textendash{} 

\item {} 
\sphinxAtStartPar
\sphinxstyleliteralstrong{\sphinxupquote{val}} (\sphinxstyleliteralemphasis{\sphinxupquote{The hex value of the desired color to change the pixel with}}) \textendash{} 

\end{itemize}

\end{description}\end{quote}

\sphinxAtStartPar
:raises TypeError : val is not type(int):
:raises ValueError : val is negative or greater than max color value (0xFFFFFF):

\end{fulllineitems}


\end{fulllineitems}



\section{Window}
\label{\detokenize{dpav:window}}

\begin{fulllineitems}
\pysiglinewithargsret{\sphinxbfcode{\sphinxupquote{class }}\sphinxcode{\sphinxupquote{dpav.window.}}\sphinxbfcode{\sphinxupquote{Window}}}{\emph{\DUrole{n}{arg1}\DUrole{p}{:} \DUrole{n}{Optional\DUrole{p}{{[}}dpav.vbuffer.VBuffer\DUrole{p}{{]}}} \DUrole{o}{=} \DUrole{default_value}{None}}, \emph{\DUrole{n}{scale}\DUrole{p}{:} \DUrole{n}{float} \DUrole{o}{=} \DUrole{default_value}{1.0}}}{}
\sphinxAtStartPar
Bases: \sphinxcode{\sphinxupquote{object}}

\sphinxAtStartPar
Handles Window capabilites of Python Direct Platform
Functions:
\begin{quote}
\begin{description}
\item[{Constructor:}] \leavevmode
\sphinxAtStartPar
\_\_init\_\_()

\item[{Setters:}] \leavevmode
\sphinxAtStartPar
set\_scale(int/float)
set\_vbuffer(VBuffer/np.ndarray,optional:int)

\item[{Getters:}] \leavevmode
\sphinxAtStartPar
get\_mouse\_pos()

\item[{Misc Methods:}] \leavevmode
\sphinxAtStartPar
open()
is\_open()
close()
update()

\item[{Private Methods:}] \leavevmode
\sphinxAtStartPar
\_update\_events(pygame.event)
\_build\_events\_dict()
\_write\_to\_screen()

\end{description}
\end{quote}


\begin{fulllineitems}
\pysigline{\sphinxbfcode{\sphinxupquote{Public}}}
\sphinxAtStartPar
vbuffer:     active VBuffer object
scale:       number that scales up/down the size of the screen
\begin{quote}

\sphinxAtStartPar
(1.0 is unscaled)
\end{quote}
\begin{description}
\item[{events:      dictionary of string:bool event pairs,}] \leavevmode\begin{description}
\item[{example:}] \leavevmode
\sphinxAtStartPar
“l\_shift”: True  \textendash{} left shift is pressed down
“l\_shift”: False \textendash{} left shift is not pressed

\end{description}

\item[{eventq:      list of active events that occured since last update cycle}] \leavevmode\begin{description}
\item[{example:}] \leavevmode
\sphinxAtStartPar
{[}‘l\_shift’, ‘mouse’, ‘a’, ‘q’{]}

\end{description}

\end{description}

\sphinxAtStartPar
debug\_flag:  boolean flag if window object should output debug info to log
open\_flag:   boolean flag for if the window is active

\end{fulllineitems}



\begin{fulllineitems}
\pysigline{\sphinxbfcode{\sphinxupquote{Private}}}~\begin{description}
\item[{\_keydict:    int:string PyGame event mapping. PyGame events identifiers are}] \leavevmode
\sphinxAtStartPar
stored as ints. This attribute is used by the public events
variable to map from PyGame’s integer:boolean pairs to
our string:boolean pairs

\item[{\_surfaces:   Two PyGame Surfaces for swapping to reflect vbuffer changes and}] \leavevmode
\sphinxAtStartPar
enable in\sphinxhyphen{}place nparray modification

\end{description}

\sphinxAtStartPar
\_screen:     PyGame.display object, used for viewing vbuffer attribute

\end{fulllineitems}



\begin{fulllineitems}
\pysiglinewithargsret{\sphinxbfcode{\sphinxupquote{close}}}{}{{ $\rightarrow$ None}}
\sphinxAtStartPar
Closes the active instance of a pygame window
\begin{quote}\begin{description}
\item[{Raises}] \leavevmode
\sphinxAtStartPar
\sphinxstyleliteralstrong{\sphinxupquote{RuntimeError}} \textendash{} no active pygame window instances exists

\end{description}\end{quote}

\end{fulllineitems}



\begin{fulllineitems}
\pysiglinewithargsret{\sphinxbfcode{\sphinxupquote{get\_mouse\_pos}}}{\emph{) \sphinxhyphen{}\textgreater{} (\textless{}class \textquotesingle{}int\textquotesingle{}\textgreater{}}, \emph{\textless{}class \textquotesingle{}int\textquotesingle{}\textgreater{}}}{}
\sphinxAtStartPar
Returns the current mouse location with respect to the pygame window instance
\begin{quote}\begin{description}
\item[{Raises}] \leavevmode
\sphinxAtStartPar
\sphinxstyleliteralstrong{\sphinxupquote{Runtime Error}} \textendash{} no active pygame window instances exists

\end{description}\end{quote}

\end{fulllineitems}



\begin{fulllineitems}
\pysiglinewithargsret{\sphinxbfcode{\sphinxupquote{is\_open}}}{}{{ $\rightarrow$ bool}}
\sphinxAtStartPar
Updates events on every call, used to abstract out PyGame
display calls and event loop
\subsubsection*{Example}
\begin{description}
\item[{if window.is\_open():}] \leavevmode
\sphinxAtStartPar
\# your code here

\end{description}
\begin{quote}\begin{description}
\item[{Returns}] \leavevmode
\sphinxAtStartPar
boolean denoting if the window is currently open

\end{description}\end{quote}

\end{fulllineitems}



\begin{fulllineitems}
\pysiglinewithargsret{\sphinxbfcode{\sphinxupquote{open}}}{}{{ $\rightarrow$ None}}
\sphinxAtStartPar
Creates and runs pygame window in a new thread

\end{fulllineitems}



\begin{fulllineitems}
\pysiglinewithargsret{\sphinxbfcode{\sphinxupquote{set\_scale}}}{\emph{\DUrole{n}{scale}\DUrole{p}{:} \DUrole{n}{float}}}{{ $\rightarrow$ None}}
\sphinxAtStartPar
Sets the window scale

\end{fulllineitems}



\begin{fulllineitems}
\pysiglinewithargsret{\sphinxbfcode{\sphinxupquote{set\_vbuffer}}}{\emph{\DUrole{n}{arg1}\DUrole{p}{:} \DUrole{n}{dpav.vbuffer.VBuffer}}}{{ $\rightarrow$ None}}
\sphinxAtStartPar
Sets the vbuffer/nparray object to display on screen
\begin{quote}\begin{description}
\item[{Parameters}] \leavevmode
\sphinxAtStartPar
\sphinxstyleliteralstrong{\sphinxupquote{arg1}} \textendash{} VBuffer/np.ndarray

\item[{Raises}] \leavevmode\begin{itemize}
\item {} 
\sphinxAtStartPar
\sphinxstyleliteralstrong{\sphinxupquote{TypeError}} \textendash{} arg1 VBuffer/np.ndarray type check

\item {} 
\sphinxAtStartPar
\sphinxstyleliteralstrong{\sphinxupquote{TypeError}} \textendash{} scale int/float type check

\end{itemize}

\end{description}\end{quote}

\end{fulllineitems}



\begin{fulllineitems}
\pysiglinewithargsret{\sphinxbfcode{\sphinxupquote{update}}}{}{{ $\rightarrow$ None}}
\sphinxAtStartPar
Pygame event abstraction, called at end of pygame loop.
Optional function if is\_open() is used
\begin{quote}\begin{description}
\item[{Raises}] \leavevmode
\sphinxAtStartPar
\sphinxstyleliteralstrong{\sphinxupquote{Runtime Error}} \textendash{} No active pygame window

\end{description}\end{quote}

\end{fulllineitems}


\end{fulllineitems}



\section{Utility}
\label{\detokenize{dpav:utility}}
\sphinxAtStartPar
The utility.py module defines a variety of utility functions to the dpav library.

\sphinxAtStartPar
This module adds utility functions for line and shape drawing, visual buffer
transformations, image parsing, and note conversions.
\subsubsection*{Examples}

\sphinxAtStartPar
\$ utility.draw\_line(vb, (3, 3), (5, 5), 0x00FF00)


\begin{fulllineitems}
\pysiglinewithargsret{\sphinxcode{\sphinxupquote{dpav.utility.}}\sphinxbfcode{\sphinxupquote{convert\_wav\_to\_nparr}}}{\emph{\DUrole{n}{wavefile}\DUrole{p}{:} \DUrole{n}{str}}}{{ $\rightarrow$ numpy.ndarray}}
\sphinxAtStartPar
Takes a string filepath of a wav file and converts it to a numpy array.

\end{fulllineitems}



\begin{fulllineitems}
\pysiglinewithargsret{\sphinxcode{\sphinxupquote{dpav.utility.}}\sphinxbfcode{\sphinxupquote{draw\_circle}}}{\emph{\DUrole{n}{vb}\DUrole{p}{:} \DUrole{n}{dpav.vbuffer.VBuffer}}, \emph{\DUrole{n}{center}\DUrole{p}{:} \DUrole{n}{list}}, \emph{\DUrole{n}{r}\DUrole{p}{:} \DUrole{n}{float}}, \emph{\DUrole{n}{color}\DUrole{p}{:} \DUrole{n}{int}}}{}
\sphinxAtStartPar
Draws a circle onto a visual buffer of a specified color and radius
around a given center point using Bresenham’s algorithm.

\end{fulllineitems}



\begin{fulllineitems}
\pysiglinewithargsret{\sphinxcode{\sphinxupquote{dpav.utility.}}\sphinxbfcode{\sphinxupquote{draw\_line}}}{\emph{\DUrole{n}{vb}\DUrole{p}{:} \DUrole{n}{dpav.vbuffer.VBuffer}}, \emph{\DUrole{n}{p0}\DUrole{p}{:} \DUrole{n}{list}}, \emph{\DUrole{n}{p1}\DUrole{p}{:} \DUrole{n}{list}}, \emph{\DUrole{n}{color}\DUrole{p}{:} \DUrole{n}{int}}}{}
\sphinxAtStartPar
Draws a line of a given color on a visual buffer from p0 to p1 using
Bresenham’s algorithm.

\end{fulllineitems}



\begin{fulllineitems}
\pysiglinewithargsret{\sphinxcode{\sphinxupquote{dpav.utility.}}\sphinxbfcode{\sphinxupquote{draw\_polygon}}}{\emph{\DUrole{n}{vb}\DUrole{p}{:} \DUrole{n}{dpav.vbuffer.VBuffer}}, \emph{\DUrole{n}{vertices}\DUrole{p}{:} \DUrole{n}{list}}, \emph{\DUrole{n}{color}\DUrole{p}{:} \DUrole{n}{int}}}{}
\sphinxAtStartPar
Draws lines of a given color connecting a list of given points in the
order they are listed

\end{fulllineitems}



\begin{fulllineitems}
\pysiglinewithargsret{\sphinxcode{\sphinxupquote{dpav.utility.}}\sphinxbfcode{\sphinxupquote{draw\_rectangle}}}{\emph{\DUrole{n}{vbuffer}\DUrole{p}{:} \DUrole{n}{dpav.vbuffer.VBuffer}}, \emph{\DUrole{n}{color}\DUrole{p}{:} \DUrole{n}{int}}, \emph{\DUrole{n}{pt1}\DUrole{p}{:} \DUrole{n}{tuple\DUrole{p}{{[}}int\DUrole{p}{, }int\DUrole{p}{{]}}}}, \emph{\DUrole{n}{pt2}\DUrole{p}{:} \DUrole{n}{tuple\DUrole{p}{{[}}int\DUrole{p}{, }int\DUrole{p}{{]}}}}}{}
\sphinxAtStartPar
Draws a rectangle into a visual buffer.
\begin{quote}\begin{description}
\item[{Parameters}] \leavevmode\begin{itemize}
\item {} 
\sphinxAtStartPar
\sphinxstyleliteralstrong{\sphinxupquote{vbuffer}} \textendash{} A visual buffer to write a rectangle into.

\item {} 
\sphinxAtStartPar
\sphinxstyleliteralstrong{\sphinxupquote{color}} \textendash{} The color the rectangle should be.

\item {} 
\sphinxAtStartPar
\sphinxstyleliteralstrong{\sphinxupquote{pt1}} \textendash{} One corder of the rectangle.

\item {} 
\sphinxAtStartPar
\sphinxstyleliteralstrong{\sphinxupquote{pt2}} \textendash{} The opposite corner from pt1 of the rectangle.

\end{itemize}

\end{description}\end{quote}
\subsubsection*{Examples}

\sphinxAtStartPar
utility.draw\_rectangle(vb, 0xFFFFFF, (3, 3), (5, 5))

\end{fulllineitems}



\begin{fulllineitems}
\pysiglinewithargsret{\sphinxcode{\sphinxupquote{dpav.utility.}}\sphinxbfcode{\sphinxupquote{fill}}}{\emph{\DUrole{n}{vb}\DUrole{p}{:} \DUrole{n}{dpav.vbuffer.VBuffer}}, \emph{\DUrole{n}{color}\DUrole{p}{:} \DUrole{n}{int}}, \emph{\DUrole{n}{vertices}}}{}
\sphinxAtStartPar
Fills a polygon defined by a set of vertices with a color.

\end{fulllineitems}



\begin{fulllineitems}
\pysiglinewithargsret{\sphinxcode{\sphinxupquote{dpav.utility.}}\sphinxbfcode{\sphinxupquote{flip\_horizontally}}}{\emph{\DUrole{n}{vb}\DUrole{p}{:} \DUrole{n}{dpav.vbuffer.VBuffer}}}{{ $\rightarrow$ dpav.vbuffer.VBuffer}}
\sphinxAtStartPar
Takes a visual buffer, flips it horizontally about the center, and
returns the new visual buffer.

\end{fulllineitems}



\begin{fulllineitems}
\pysiglinewithargsret{\sphinxcode{\sphinxupquote{dpav.utility.}}\sphinxbfcode{\sphinxupquote{flip\_vertically}}}{\emph{\DUrole{n}{vb}\DUrole{p}{:} \DUrole{n}{dpav.vbuffer.VBuffer}}}{{ $\rightarrow$ dpav.vbuffer.VBuffer}}
\sphinxAtStartPar
Takes a visual buffer, flips it vertically about the center, and returns
the new visual buffer.

\end{fulllineitems}



\begin{fulllineitems}
\pysiglinewithargsret{\sphinxcode{\sphinxupquote{dpav.utility.}}\sphinxbfcode{\sphinxupquote{get\_note\_from\_string}}}{\emph{\DUrole{n}{note}\DUrole{p}{:} \DUrole{n}{str}}, \emph{\DUrole{n}{octave}\DUrole{p}{:} \DUrole{n}{int}}}{{ $\rightarrow$ int}}
\sphinxAtStartPar
Converts a string denoting a note and an octave into a frequency.
\begin{quote}\begin{description}
\item[{Parameters}] \leavevmode
\sphinxAtStartPar
\sphinxstyleliteralstrong{\sphinxupquote{note}} \textendash{} A musical note denoted with a capital letter and a
sharp (\#) or a flat (b).

\item[{Returns}] \leavevmode
\sphinxAtStartPar
A frequency in hertz.

\end{description}\end{quote}

\end{fulllineitems}



\begin{fulllineitems}
\pysiglinewithargsret{\sphinxcode{\sphinxupquote{dpav.utility.}}\sphinxbfcode{\sphinxupquote{load\_image}}}{\emph{\DUrole{n}{filepath}\DUrole{p}{:} \DUrole{n}{str}}}{{ $\rightarrow$ numpy.ndarray}}
\sphinxAtStartPar
Converts an image and returns a numpy array representation of
that image in hex.
\begin{quote}\begin{description}
\item[{Parameters}] \leavevmode
\sphinxAtStartPar
\sphinxstyleliteralstrong{\sphinxupquote{filepath}} \textendash{} The filepath of the image to be loaded

\item[{Returns}] \leavevmode
\sphinxAtStartPar
A numpy array filled with the hex color data of the image

\end{description}\end{quote}

\end{fulllineitems}



\begin{fulllineitems}
\pysiglinewithargsret{\sphinxcode{\sphinxupquote{dpav.utility.}}\sphinxbfcode{\sphinxupquote{point\_in\_polygon}}}{\emph{\DUrole{n}{x}\DUrole{p}{:} \DUrole{n}{int}}, \emph{\DUrole{n}{y}\DUrole{p}{:} \DUrole{n}{int}}, \emph{\DUrole{n}{vertices}}}{{ $\rightarrow$ bool}}
\sphinxAtStartPar
Uses the Even\sphinxhyphen{}Odd Rule to determien whether or not a given pixel is inside
a given set of vertices.
\begin{quote}\begin{description}
\item[{Parameters}] \leavevmode\begin{itemize}
\item {} 
\sphinxAtStartPar
\sphinxstyleliteralstrong{\sphinxupquote{x}} \textendash{} The x coordinate of the pixel to be checked.

\item {} 
\sphinxAtStartPar
\sphinxstyleliteralstrong{\sphinxupquote{y}} \textendash{} The y coordinate of the pixel to be checked.

\end{itemize}

\item[{Returns}] \leavevmode
\sphinxAtStartPar
True if the pixel is within the polygon, False otherwise.

\end{description}\end{quote}

\end{fulllineitems}



\begin{fulllineitems}
\pysiglinewithargsret{\sphinxcode{\sphinxupquote{dpav.utility.}}\sphinxbfcode{\sphinxupquote{replace\_color}}}{\emph{\DUrole{n}{vb}\DUrole{p}{:} \DUrole{n}{dpav.vbuffer.VBuffer}}, \emph{\DUrole{n}{replaced\_color}\DUrole{p}{:} \DUrole{n}{int}}, \emph{\DUrole{n}{new\_color}\DUrole{p}{:} \DUrole{n}{int}}}{}
\sphinxAtStartPar
Replaces all pixels in a visual buffer of a chosen color with a new
color.

\end{fulllineitems}



\begin{fulllineitems}
\pysiglinewithargsret{\sphinxcode{\sphinxupquote{dpav.utility.}}\sphinxbfcode{\sphinxupquote{rgb\_to\_hex}}}{\emph{\DUrole{n}{arr}\DUrole{p}{:} \DUrole{n}{numpy.ndarray}}}{{ $\rightarrow$ numpy.ndarray}}
\sphinxAtStartPar
Converts a numpy array with (r, g, b) values into a numpy array with
hex color values.

\end{fulllineitems}



\begin{fulllineitems}
\pysiglinewithargsret{\sphinxcode{\sphinxupquote{dpav.utility.}}\sphinxbfcode{\sphinxupquote{translate}}}{\emph{\DUrole{n}{vb}\DUrole{p}{:} \DUrole{n}{dpav.vbuffer.VBuffer}}, \emph{\DUrole{n}{x\_translation}\DUrole{p}{:} \DUrole{n}{int}}, \emph{\DUrole{n}{y\_translation}\DUrole{p}{:} \DUrole{n}{int}}}{{ $\rightarrow$ dpav.vbuffer.VBuffer}}
\sphinxAtStartPar
Takes a visual buffer, translates every pixel in it by given values, and
returns the new visual buffer

\end{fulllineitems}




\renewcommand{\indexname}{Index}
\printindex
\end{document}